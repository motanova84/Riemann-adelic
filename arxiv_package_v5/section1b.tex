We work in the ring of adeles $\mathbb{A}_\mathbb{Q}$ with the Haar measure
$dx$ and the multiplicative measure $d^{\times}x$. We denote by
$\mathcal{S}(\mathbb{A}_\mathbb{Q})$ the Schwartz--Bruhat space
\cite[Chap.~I]{tate1967}.

The objective of this section is to demonstrate that the conditions A1--A4 introduced
for the construction of $D(s)$ are not independent axioms, but necessary consequences
of the classical adelic formalism.

\begin{theorem}[A1: finite-scale flow]\label{thm:A1}
Let $\Phi\in\mathcal{S}(\mathbb{A}_\mathbb{Q})$ be factorizable as
$\Phi=\prod_v \Phi_v$.
Then the scale flow $u\mapsto \Phi(u\cdot)$ has finite energy and
discrete orbits with lengths $\ell_v = \log q_v$.
\end{theorem}

\begin{proof}
Each $\Phi_v$ is Gaussian in $\mathbb{R}$ or compact in $\mathbb{Q}_p$.
Let $U\subset \mathbb{A}_\mathbb{Q}^\times$ be compact. Then
\[
 \int_U \!\int_{\mathbb{A}_\mathbb{Q}} |\Phi(ux)|^2\,dx\,d^\times u
   = \prod_v \int_{U_v} \!\int_{\mathbb{Q}_v} |\Phi_v(u_v x_v)|^2\,dx_v\,d^\times u_v.
\]
In $\mathbb{R}$, the Gaussian decay gives uniform integrability;
in $\mathbb{Q}_p$, compactness ensures finite measure.
The orbits are discrete and their length is $\ell_v=\log q_v$, derived from the
local multiplicative norm.
\end{proof}

\begin{theorem}[A2: functional symmetry]\label{thm:A2}
Let $Z(\Phi,s)$ be the Tate zeta-integral associated to $\Phi$.
Then the completed function
$D(s)=\Gamma_\mathbb{A}(s)Z(\Phi,s)$ satisfies
\[
 D(1-s)=D(s).
\]
\end{theorem}

\begin{proof}
The adelic Poisson identity
\cite[Thm.~2]{tate1967} implies
$Z(\widehat{\Phi},1-s)=Z(\Phi,s)$ if the local transforms are normalized with
the Weil factors $\gamma_v(s)$.
The product law $\prod_v \gamma_v(s)=1$ \cite[§II.3]{Weil1964}
ensures global symmetry.
Therefore $D(s)=D(1-s)$.
\end{proof}

\begin{theorem}[A4: spectral regularity]\label{thm:A4}
Let $K_s$ be the integral kernel
\[
 K_s(x,y)=\Phi(x)\overline{\Phi(y)}|xy^{-1}|_\mathbb{A}^{s-1/2}
\]
with $\Phi\in\mathcal{S}(\mathbb{A}_\mathbb{Q})$.
Then the operator $T_s f(x)=\int_{\mathbb{A}_\mathbb{Q}} K_s(x,y)f(y)\,dy$
is trace-class, depends holomorphically on $s$ in vertical strips, and its spectrum
is discrete and continuous in $s$.
\end{theorem}

\begin{proof}
For $\Re(s)=\tfrac12$, $K_s\in L^2(\mathbb{A}_\mathbb{Q}^2)$, so that
$T_s$ is Hilbert--Schmidt.
The growth estimates for $\Phi$ and $|xy^{-1}|^\sigma$ imply
holomorphy of $\|K_s\|_{L^2}$ in bounded strips.
The Birman--Solomyak theorem \cite[Thm.~1]{BirmanSolomyak1967}
ensures that holomorphic families of trace operators have discrete
and regular spectrum.
Thus, A4 is a direct consequence of the formalism.
\end{proof}

\bigskip
With these results, A1, A2 and A4 are \textbf{proven} within the classical adelic
framework, and cease to be independent axioms.