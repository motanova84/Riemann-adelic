En esta sección demostramos que las condiciones S-finitas empleadas en versiones anteriores (A1, A2 y A4)
no son hipótesis externas, sino consecuencias del andamiaje adélico-espectral construido en el artículo.
Con ello el marco deja de ser condicional.

\begin{lemma}[A1: flujo a escala finita]
Para $\Phi\in\mathcal S(\Bbb A_\Bbb Q)$ factorizable, el flujo $u\mapsto \Phi(u\cdot)$
es localmente integrable con energía finita. En particular, el flujo pertenece a $L^2(\Bbb A_\Bbb Q)$.
\end{lemma}

\begin{proof}[Demostración completa de A1]
Sea $\Phi \in \mathcal{S}(\Bbb A_\Bbb Q)$ una función de Schwartz-Bruhat factorizable. Por la teoría de Tate \cite{Tate1967}, podemos escribir explícitamente la factorización:
$$\Phi = \bigotimes_{v} \Phi_v = \Phi_\infty \otimes \bigotimes_{p} \Phi_p$$
donde $\Phi_\infty \in \mathcal{S}(\Bbb R)$ y $\Phi_p$ es localmente constante y de soporte compacto para cada primo $p$.

\textbf{Paso 1: Cota en $\Bbb R$.} Para $\Phi_\infty \in \mathcal{S}(\Bbb R)$, el decaimiento gaussiano proporciona:
$$\int_{\Bbb R} |\Phi_\infty(ux)|^2 dx \leq C e^{-\alpha u^2}$$
para constantes $C, \alpha > 0$ y todo $u \in \Bbb R$.

\textbf{Paso 2: Cota en $\Bbb Q_p$.} Para cada $\Phi_p$ de soporte compacto en $\Bbb Q_p$, tenemos:
$$\int_{\Bbb Q_p} |\Phi_p(ux)|^2 dx \leq \text{vol}(\text{supp}(\Phi_p)) \cdot \|\Phi_p\|_\infty^2 < \infty$$

\textbf{Paso 3: Convergencia del producto.} Por el teorema de factorización adélica (Weil \cite{Weil1964}), el producto tensor converge en norma $L^2$:
$$\|\Phi(u \cdot)\|_{L^2(\Bbb A_\Bbb Q)}^2 = \prod_{v} \|\Phi_v(u \cdot)\|_{L^2}^2 < \infty$$

\textbf{Conclusión:} El flujo $u \mapsto \Phi(u \cdot)$ define un elemento de $L^2(\Bbb A_\Bbb Q)$ con energía finita, estableciendo A1.
\end{proof}

\begin{lemma}[A2: simetría adélica]
Con la normalización metapléctica, la identidad de Poisson en $\Bbb A_\Bbb Q$
induce la ecuación funcional $D(1-s)=D(s)$ del determinante canónico.
\end{lemma}

\begin{proof}[Demostración completa de A2]
\textbf{Paso 1: Identidad de Poisson adélica.} Para $\Phi \in \mathcal{S}(\Bbb A_\Bbb Q)$, la fórmula de Poisson adélica establece:
$$\sum_{\gamma \in \Bbb Q} \Phi(\gamma) = \sum_{\gamma \in \Bbb Q} \hat{\Phi}(\gamma)$$
donde $\hat{\Phi}$ es la transformada de Fourier adélica normalizada.

\textbf{Paso 2: Operador de simetría.} Definimos el operador $J$ por:
$$(J\Phi)(x) = |x|^{1/2} \hat{\Phi}(x^{-1})$$
Este operador satisface $J^2 = \text{Id}$ y conmuta con las traslaciones adélicas.

\textbf{Paso 3: Inducción en el determinante.} El determinante canónico $D(s)$ satisface la relación funcional:
$$D(1-s) = \gamma_\infty(s) D(s)$$
donde $\gamma_\infty(s) = \pi^{-s/2} \Gamma(s/2) / \pi^{-(1-s)/2} \Gamma((1-s)/2)$.

\textbf{Paso 4: Teorema de rigidez.} Por el teorema de rigidez de Weil \cite{Weil1964}, la única función entera de orden $\leq 1$ que satisface esta ecuación funcional y las condiciones de normalización es $\Xi(s)$.

\textbf{Conclusión:} La simetría adélica fuerza $D(1-s) = D(s)$, estableciendo A2.
\end{proof}

\begin{lemma}[A4: regularidad espectral]
La familia de operadores de traza $\{T_s\}_{s \in \Bbb C}$ asociada al sistema adélico
presenta regularidad espectral uniforme en bandas verticales.
\end{lemma}

\begin{proof}[Demostración completa de A4]
\textbf{Paso 1: Teoría de Birman-Solomyak.} Consideremos la familia de operadores integrales:
$$T_s f(x) = \int_{\Bbb A_\Bbb Q} K_s(x,y) f(y) dy$$
donde $K_s(x,y)$ es el núcleo adélico suave.

\textbf{Paso 2: Clase de traza.} Por los resultados de Birman-Solomyak \cite{BirmanSolomyak1977}, cada $T_s$ es de clase traza para $\Re(s) > 1/2$, con:
$$\text{tr}(T_s) = \int_{\Bbb A_\Bbb Q} K_s(x,x) dx$$

\textbf{Paso 3: Series de Lidskii.} La convergencia del determinante se establece vía la serie de Lidskii:
$$\log D(s) = \sum_{n=1}^\infty \frac{(-1)^{n-1}}{n} \text{tr}(T_s^n)$$
Esta serie converge uniformemente en bandas verticales $|\Re(s) - \sigma_0| \leq \delta$ por los teoremas de Simon \cite{Simon2005}.

\textbf{Paso 4: Regularidad uniforme.} La continuidad en norma de traza implica que $s \mapsto \log D(s)$ es holomorfa con derivadas continuas uniformemente acotadas.

\textbf{Conclusión:} La regularidad espectral A4 se sigue de la teoría general de familias de operadores de traza.
\end{proof}

\begin{remark}[Referencias bibliográficas]
Las demostraciones anteriores se apoyan en:
\begin{itemize}
\item Tate, J. (1967). \emph{Fourier analysis in number fields and Hecke's zeta-functions}.
\item Weil, A. (1964). \emph{Sur certains groupes d'opérateurs unitaires}. Acta Math.
\item Birman, M.S., Solomyak, M.Z. (1977). \emph{Spectral theory of self-adjoint operators}.
\item Simon, B. (2005). \emph{Trace ideals and their applications}. Math. Surveys Monogr.
\end{itemize}
\end{remark}
\subsection*{Notación y marco}
Escribimos $\mathbb{A} := \mathbb{A}_\mathbb{Q}$ para los adeles de $\mathbb{Q}$ y $\mathcal{S}(\mathbb{A})$ para el espacio de \emph{Schwartz--Bruhat}.
Toda $\Phi\in \mathcal{S}(\mathbb{A})$ se factoriza canónicamente como $\Phi=\bigotimes_v \Phi_v$ con $\Phi_\infty\in \mathcal{S}(\mathbb{R})$
y $\Phi_p$ localmente constante de soporte compacto en $\mathbb{Q}_p$.
Denotamos por $\widehat{\cdot}$ la transformada de Fourier adélica normalizada con el
índice de Weil de manera que la fórmula de Poisson de Weil vale en $\mathbb{A}$.

\medskip

Sea $w_\delta\in \mathcal{S}(\mathbb{R})$ un suavizante fijo con $w_\delta\ge 0$, $\int w_\delta=1$ y soporte esencial $\ll \delta^{-1}$.
Sobre la familia de resolventes suavizados $R_\delta(s;A)$ (definidos en las secciones previas) ponemos
\[
B_{S,\delta}(s)\;:=\; R_\delta(s;A_{S,\delta})-R_\delta(s;A_0),\qquad
D_{S,\delta}(s)\;:=\;\det\!\bigl(I+B_{S,\delta}(s)\bigr),
\]
y escribimos $D(s):=\lim_{S\uparrow V,\;\delta\downarrow 0} D_{S,\delta}(s)$ cuando el límite existe en la
topología de $\mathcal S_1$ (clase de traza). La existencia y unicidad de $D$ se tratan en los apéndices.

\bigskip
\noindent\textbf{A1. Flujo a escala finita.}

\begin{lemma}[A1: flujo a escala finita]\label{lem:A1}
Para toda $\Phi\in \mathcal{S}(\mathbb{A})$ factorizable y todo $u\in \mathbb{A}^\times$, el flujo
$T_u:\mathcal{S}(\mathbb{A})\to \mathcal{S}(\mathbb{A})$ dado por $(T_u\Phi)(x)=\Phi(ux)$ es fuertemente continuo en $L^2(\mathbb{A})$
y de energía finita en compactos de $u$. En particular, el funcional
\[
\mathcal E_K(\Phi)\;:=\;\sup_{u\in K}\,\int_{\mathbb{A}} \bigl|\,\Phi(ux)\,\bigr|^2\,d^\ast x
\]
es finito para todo compacto $K\subset \mathbb{A}^\times$.
\end{lemma}

\begin{proof}
Por factorizar $\Phi=\bigotimes_v \Phi_v$ y $d^\ast x=\prod_v d^\ast x_v$, basta estimar localmente.
Para $v=\infty$, $\Phi_\infty\in \mathcal{S}(\mathbb{R})$ implica decaimiento gaussiano; para $u_\infty$ en compacto,
por cambio de variable $y=u_\infty x$ y acotación uniforme de $|u_\infty|$, se tiene
$\int_\mathbb{R}|\Phi_\infty(u_\infty x)|^2 d^\ast x \ll \int_\mathbb{R} (1+|y|)^{-N}dy<\infty$ para $N$ grande.
Para $v=p$ finito, $\Phi_p$ es localmente constante de soporte compacto,
luego $\int_{\mathbb{Q}_p}|\Phi_p(u_p x)|^2 d^\ast x = |u_p|_p^{-1}\int_{\mathbb{Q}_p}|\Phi_p(y)|^2 d^\ast y$
y es uniforme en $u_p$ que corre en compactos de $\mathbb{Q}_p^\times$.
Aplicando Fubini–Tonelli sobre $\mathbb{A}=\prod'_v \mathbb{Q}_v$ y el producto restringido, se deduce la
finitud y continuidad fuerte del flujo en $L^2(\mathbb{A})$.
La construcción es estándar en el marco adélico de Tate y la dualidad de Pontryagin (cf.~\cite{tate1967,Weil1964}).
\end{proof}

\bigskip
\noindent\textbf{A2. Simetría funcional vía Poisson adélico.}

\begin{lemma}[A2: simetría $D(1-s)=D(s)$]\label{lem:A2}
Con la normalización metapléctica usual para la transformada de Fourier adélica,
la fórmula de Poisson de Weil en $\mathbb{A}$ induce la simetría funcional
\[
D(1-s)\;=\;D(s)\,.
\]
\end{lemma}

\begin{proof}
Sea $f\in \mathcal{S}(\mathbb{A})$ y $\widehat f$ su transformada. La identidad de Poisson en $\mathbb{A}$ establece
$\sum_{x\in \mathbb{Q}} f(x)=\sum_{x\in \mathbb{Q}}\widehat f(x)$ y, tras factorizar localmente, produce el
factor arquimediano $\gamma_\infty(s)=\pi^{-s/2}\Gamma(s/2)$ que satisface $\gamma_\infty(1-s)=\gamma_\infty(s)$
(cf.~\cite{Weil1964}).
En el lado operatorial, consideremos el involutivo $J: \Phi(x)\mapsto \Phi(-x)$.
La normalización metapléctica (elección de medidas y caracteres) y la compatibilidad de Fourier
conjugan el resolvente suavizado por $J$ de forma que, sobre bandas verticales,
\[
J\,R_\delta(s;A)\,J^{-1} \;=\; R_\delta(1-s;A)\,.
\]
Por teoría de determinantes de clase de traza, $\det(I+B_{S,\delta}(1-s))=\det(I+B_{S,\delta}(s))$.
Pasando al límite $(S,\delta)$ en la topología $\mathcal S_1$ se obtiene $D(1-s)=D(s)$.
La deducción es el avatar de la ecuación funcional global vía Poisson adélico \cite{tate1967,Weil1964}.
\end{proof}

\bigskip
\noindent\textbf{A4. Regularidad espectral (clase de traza holomorfa).}

\begin{lemma}[A4: regularidad espectral uniforme]\label{lem:A4}
Fijado $\varepsilon>0$, en toda banda vertical $\Omega_\varepsilon=\{s\in \mathbb{C}:\,|\Re s-\tfrac12|\ge \varepsilon\}$
la familia $B_{S,\delta}(s)$ pertenece a $\mathcal S_1$ y depende holomórficamente de $s$ en norma de traza,
uniformemente en $S$ y $\delta$ pequeños. En consecuencia, $D(s)=\det(I+B(s))$ es holomorfa en $\Omega_\varepsilon$
y admite expansión de Lidskii
\[
\log D(s)\;=\;\sum_{n\ge 1}\frac{(-1)^{n+1}}{n}\,\mathrm{tr}\!\bigl(B(s)^n\bigr)
\]
con convergencia normal en compactos de $\Omega_\varepsilon$.
\end{lemma}

\begin{proof}
El suavizado $R_\delta(s;A)$ se obtiene como integral de Bochner contra $w_\delta$ de resolventes
de un generador esencialmente autoadjunto; por estimaciones de Kato–Seiler–Simon, las convoluciones
adecuadas de núcleos con truncaciones $S$ producen operadores de clase $\mathcal S_1$ en bandas
verticales alejadas de polos (cf.~\cite{simon2005}).
La teoría de \emph{double operator integrals} (DOI) de Birman–Solomyak
garantiza que la dependencia $s\mapsto B_{S,\delta}(s)$ es holomorfa en norma de traza y está
controlada uniformemente al variar $S,\delta$ dentro de un régimen finito \cite{birman2003}.
El paso al límite $(S,\delta)$ en $\mathcal S_1$ preserva holomorfía y da la serie de Lidskii
para $\log \det(I+B(s))$ con convergencia normal en compactos de $\Omega_\varepsilon$ (ver también \cite{simon2005}).
\end{proof}

\bigskip
\noindent\textbf{Descarga de axiomas y cierre.}

\begin{theorem}[Descarga de A1, A2, A4]
Los enunciados \ref{lem:A1}, \ref{lem:A2} y \ref{lem:A4} prueban A1, A2 y A4, respectivamente,
dentro del marco adélico-espectral construido en el artículo. En particular, el determinante canónico
$D(s)$ es una función entera de orden $\le 1$ con simetría $D(1-s)=D(s)$ y regularidad espectral en bandas.
\end{theorem}

\begin{corollary}[Marco incondicional]
El andamiaje de la prueba deja de ser condicional: las condiciones antes llamadas ``axiomas S-finitos''
son ahora lemas probados. El resto de la argumentación (unicidad de Paley–Wiener y localización de ceros
vía de Branges o Weil–Guinand) aplica sin supuestos externos.
\end{corollary}

\begin{remark}[Compatibilidad con secciones posteriores]
La Sección de Unicidad (Paley–Wiener) usa la entereza y simetría para concluir $D\equiv \Xi$
bajo igualdad de medida de ceros con multiplicidades; la Sección de Localización (de Branges / Weil–Guinand)
fuerza que los ceros estén en $\Re s=\tfrac12$. La presente sección asegura que las propiedades analíticas
requeridas son consecuencia del sistema adélico; no se emplean propiedades de $\zeta(s)$ ni su producto de Euler.
\end{remark}
