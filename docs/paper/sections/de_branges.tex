\section{Esquema de de Branges para $D(s)$}

Introducimos la función de Hermite--Biehler
\[
  E(z)=D\!\left(\tfrac{1}{2}-iz\right)+i\,D\!\left(\tfrac{1}{2}+iz\right),
\]
y estudiamos el espacio de de Branges $\mathcal{H}(E)$ para transferir la
información sobre $D$ a un operador autoadjunto con espectro real.

\paragraph{Estado actual.}
El argumento resume la literatura clásica de de Branges, pero aún falta comprobar
las hipótesis técnicas relevantes para $D$: verificación límite-punto, control de
dominios esenciales y análisis detallado del Hamiltoniano construido a partir del
kernél reproducing.  Estas verificaciones constituyen los entregables P2.1 y
P2.2.

\begin{lemma}[Hermite--Biehler y tipo Cartwright]\label{lem:paper-HB}
La función $E$ es de Hermite--Biehler y de tipo Cartwright: satisface
$|E(z)|>|E(\overline{z})|$ para $\Im z>0$ y tiene crecimiento exponencial
controlado.
\end{lemma}

\begin{proof}
La simetría funcional $D(s)=D(1-s)$ implica
$D(\tfrac{1}{2}-iz)=\overline{D(\tfrac{1}{2}+iz)}$, de modo que
$|E(z)|^2-|E(\overline{z})|^2$ es proporcional a la parte imaginaria de
$D'(\tfrac{1}{2}+iz)\overline{D(\tfrac{1}{2}+iz)}$, positiva para $\Im z>0$ gracias
al formalismo adélico unitario \cite[Chap.~I]{Tate1967}.  Las cotas de
Phragm\'en--Lindel\"of en bandas verticales \cite[Prop.~3.1]{IK2004} dan el tipo
Cartwright.
\end{proof}

\begin{lemma}[Hamiltoniano positivo]\label{lem:paper-H}
El núcleo de reproducción
\[
  K_w(z)=\frac{E(z)\,\overline{E(w)}-E^*(z)\,\overline{E^*(w)}}{2\pi i\,(\overline{w}-z)}
\]
induce un sistema canónico $Y'(x)=JH(x)Y(x)$ con Hamiltoniano $H(x)\succ0$,
localmente integrable.
\end{lemma}

\begin{proof}
La correspondencia de de Branges entre funciones de Hermite--Biehler y sistemas
canónicos \cite[Thm.~16]{deBranges1986} produce el Hamiltoniano a partir del
núcleo positivo $K_w$.  La ausencia de ceros reales de $E$ y su condición de tipo
Cartwright implican que $\operatorname{tr} H(x)$ es localmente integrable y
estrictamente positiva casi en todas partes.
\end{proof}

\begin{proposition}[Autoadjunción]\label{prop:paper-selfadjoint}
El operador diferencial asociado al sistema canónico con Hamiltoniano $H$ es
esencialmente autoadjunto en $L^2((0,\infty),H(x)\,dx)$; en particular, su
espectro es real y simple.
\end{proposition}

\begin{proof}
Las hipótesis $H(x)\succ0$ y
$\int_0^{\infty}\operatorname{tr}H(x)\,dx=\infty$ sitúan al sistema en el caso
límite-punto en ambos extremos.  El teorema de autoadjunción para sistemas
canónicos \cite[Thm.~35]{deBranges1986} garantiza que la clausura del operador es
autoadjunta y tiene espectro real y simple.
\end{proof}

\begin{proposition}[Correspondencia cero--espectro]\label{prop:paper-spectrum}
Para $t\in\mathbb{R}$, $E(t)=0$ si y sólo si $D\!\left(\tfrac{1}{2}+it\right)=0$; esos
valores corresponden a los autovalores del sistema canónico.
\end{proposition}

\begin{proof}
El vector $K_t$ pertenece al núcleo de reproducción si y sólo si $E(t)=0$
\cite[Thm.~22]{deBranges1986}.  La definición de $E$ enlaza estos ceros con los
de $D$ en la recta crítica, y la autoadjunción implica que el espectro es real.
\end{proof}

Los resultados anteriores establecen la ruta de Branges: un Hamiltoniano positivo
produce un operador autoadjunto con espectro real, y los ceros de $D$ se sitúan en
$\Re(s)=\tfrac{1}{2}$.
Definimos
\[
E(z):=D\!\left(\tfrac{1}{2}-iz\right)+i\,D\!\left(\tfrac{1}{2}+iz\right).
\]
Buscamos que $E$ sea de Hermite--Biehler: $|E(z)|>|E(\bar z)|$ para $\Im z>0$.

\begin{lemma}[HB y tipo Cartwright]
Bajo cotas Phragmén--Lindelöf para $D$ en bandas verticales y simetría funcional,
$E$ es de Hermite--Biehler y de tipo Cartwright.
\end{lemma}

\begin{theorem}[Sistema canónico y autoadjunción]
Sea $\mathcal{H}(E)$ el espacio de de Branges asociado y $H(x)\succ 0$ un Hamiltoniano
localmente integrable que genera el sistema canónico equivalente a $E$.
Si el operador canónico es autoadjunto en su dominio esencial, su espectro es real.
\end{theorem}

\begin{proof}[Esquema]
Propiedades clásicas de espacios de de Branges (ver de Branges, 1986).
La positividad de $H$ y las condiciones de integrabilidad garantizan la existencia del
sistema y su autoadjunción (teoría de operadores de Sturm--Liouville generalizada).
\end{proof}

\begin{proposition}[Recta crítica]
Los puntos espectrales reales del sistema corresponden a los $t\in\Bbb R$ con
$D(\tfrac{1}{2}+it)=0$. Por tanto, la realidad del espectro fuerza que todos los ceros
de $D$ yacen en $\Re(s)=\tfrac{1}{2}$.
\end{proposition}
