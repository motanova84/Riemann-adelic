The Riemann Hypothesis stands as one of the most profound unsolved problems in mathematics, asserting that all non-trivial zeros of the Riemann zeta function $\zeta(s)$ lie on the critical line $\Re(s) = 1/2$. Despite numerous attempts spanning over 160 years, a complete proof has remained elusive.

This paper presents a novel approach based on \emph{S-finite adelic spectral systems}—a framework that emerges from the intersection of spectral theory, adelic analysis, and operator theory. Rather than beginning with the classical Euler product definition of $\zeta(s)$, we construct a canonical determinant $D(s)$ from first principles using operator-theoretic methods.

\subsection{Main Contributions}

Our principal contributions are threefold:

\begin{enumerate}
\item \textbf{Canonical Construction}: We define $D(s)$ via a scale-invariant flow over abstract places, smoothed through double operator integrals, without presupposing the structure of $\zeta(s)$.

\item \textbf{Spectral Framework}: We establish that $D(s)$ satisfies the key analytical properties (entire function of order $\leq 1$, functional equation, normalization) and prove its identification with the completed Riemann xi-function $\Xi(s)$.

\item \textbf{Dual Proof Strategy}: We employ both the de Branges theory of Hilbert spaces of entire functions and the Weil-Guinand positivity methods to establish that zeros must lie on the critical line.
\end{enumerate}

\subsection{Structure of the Paper}

The paper is organized as follows: Section 2 establishes the axiomatic foundation and spectral system. Sections 3-7 develop the core theoretical framework, including archimedean rigidity, uniqueness theorems, and the critical localization result. The appendices provide detailed technical proofs and numerical validation.

All computational code and data are made available in the associated GitHub repository for full transparency and reproducibility.