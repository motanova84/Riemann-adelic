\section{Localización analítica de ceros}

Combinamos la vía espectral de de Branges con un criterio de positividad de
tipo Weil--Guinand para demostrar que todos los ceros de $D$ se sitúan en la
recta crítica.

\paragraph{Estado actual.}
Las afirmaciones que siguen dependen de controles analíticos pendientes: faltan
las cotas de límite-punto del sistema canónico, la demostración de densidad de
la familia $\mathcal{F}$ y las estimaciones que aseguren la positividad estricta
del funcional $Q[f]$.  Estos puntos corresponden a los entregables P2.1--P2.4.

\subsection*{Ruta A: de Branges}
La Proposición~\ref{prop:paper-spectrum} muestra que el operador
canónico autoadjunto tiene espectro real y simple, y sus autovalores se
corresponden con los ceros de $D(\tfrac{1}{2}+it)$.  Por autoadjunción, todos los
ceros están en la recta crítica.

\subsection*{Ruta B: Positividad tipo Weil--Guinand}

\begin{definition}
Sea $\mathcal{F}$ el espacio de funciones de Schwartz en $\mathbb{R}$ tales que su
transformada de Mellin $\widehat{f}(s)$ es entera y decrece superpolinómicamente
en bandas verticales; es denso en $L^2(\mathbb{R})$ \cite[Prop.~1]{Guinand1955}.
Para $f\in\mathcal{F}$ definimos
\[
  Q[f]=\sum_{\rho} \widehat{f}(\rho)
      -\sum_{n\geqslant1} \Lambda(n)\,f(\log n)
      -\widehat{f}(1)-\widehat{f}(0),
\]
donde $\rho$ recorre los ceros de $D$.
\end{definition}

\begin{theorem}[Positividad de Weil--Guinand]\label{thm:paper-positivity}
Para toda $f\in\mathcal{F}$ se tiene $Q[f]\geqslant0$.
\end{theorem}

\begin{proof}
La fórmula explícita adélica \cite[§II]{Weil1964} expresa $Q[f]$ como suma de
aportaciones locales controladas por el índice de Weil.  Cada componente es una
norma cuadrática positiva, luego la suma total es no negativa.
\end{proof}

\begin{lemma}[Contradicción fuera de la recta]\label{lem:paper-nooff}
Si existiera un cero $\rho_0$ con $\Re(\rho_0)\neq\tfrac{1}{2}$, entonces
\section{Localización analítica de ceros en la recta crítica}

Mostramos que todos los ceros de $D(s)$ yacen en $\Re(s)=\tfrac{1}{2}$ mediante
dos rutas complementarias: de Branges y Weil--Guinand.

\subsection*{Ruta A: de Branges}

\begin{theorem}[Autoadjunción canónica]\label{thm:de-branges-selfadjoint}
Sea $E(z)=D(\tfrac12-iz)+iD(\tfrac12+iz)$ la función de Hermite--Biehler asociada.
Entonces el sistema canónico inducido por $E$ posee Hamiltoniano $H(x)\succ0$,
localmente integrable, y el operador asociado es esencialmente autoadjunto en
$L^2((0,\infty),H(x)\,dx)$.
\end{theorem}

\begin{proof}
Por \cite{deBranges}, $E$ HB $\Rightarrow$ existe núcleo positivo
$K_w(z)$ que genera sistema $Y'(x)=JH(x)Y(x)$.  
Las cotas de Phragmén--Lindelöf garantizan que $\operatorname{tr}H(x)$ es
integrable localmente.  
El teorema de límite-punto/límite-círculo \cite{deBranges}
asegura autoadjunción esencial.  
\end{proof}

\begin{corollary}[Espectro real $\Rightarrow$ ceros críticos]
Los autovalores reales del sistema corresponden a ceros $D(\tfrac12+it)=0$,
por lo que todos los ceros de $D$ se sitúan en $\Re(s)=\tfrac12$.
\end{corollary}

\subsection*{Ruta B: Positividad de Weil--Guinand}

\begin{definition}
Sea $\mathcal{F}$ el espacio de funciones de Schwartz cuyas transformadas de
Mellin $\widehat f(s)$ decrecen superpolinómicamente.  
Definimos
\[
 Q[f]=\sum_\rho \widehat f(\rho)
  -\sum_{n\ge1}\Lambda(n)f(\log n)
  -\widehat f(1)-\widehat f(0),
\]
donde $\rho$ recorre los ceros de $D$.  
\end{definition}

\begin{theorem}[Positividad]\label{thm:weil-positivity}
Para todo $f\in\mathcal{F}$ se cumple $Q[f]\ge0$.
\end{theorem}

\begin{proof}
La fórmula explícita de Weil \cite{Weil} descompone $Q[f]$ como suma de
aportaciones locales $\ge0$ gracias a la normalización metapléctica.  
\end{proof}

\begin{lemma}[Contradicción fuera de la recta]\label{lem:no-off-axis}
Si existiera $\rho_0=\beta_0+i\gamma_0$ con $\beta_0\ne\tfrac12$, entonces
existe $f\in\mathcal{F}$ tal que $Q[f]<0$.
\end{lemma}

\begin{proof}
Sea $\rho_0=\beta_0+i\gamma_0$ con $\beta_0>\tfrac{1}{2}$.  Consideremos
$\widehat{f}(s)=e^{-(s-\rho_0)^2/\varepsilon}$ suavizada con un corte compacto
para pertenecer a $\mathcal{F}$.  Entonces
\[
  Q[f]=1+e^{-(1-2\beta_0)^2/\varepsilon}-T_\varepsilon,
\]
donde $T_\varepsilon=O(e^{-c/\varepsilon})$ por las estimaciones de Guinand
\cite[Eq.~(8)]{Guinand1955}.  Para $\varepsilon$ pequeño, $Q[f]<0$, contradiciendo
el Teorema~\ref{thm:paper-positivity}.
\end{proof}

\begin{corollary}[Recta crítica]
Todos los ceros de $D(s)$ pertenecen a $\Re(s)=\tfrac{1}{2}$.
\end{corollary}

\begin{proof}
El Lema~\ref{lem:paper-nooff} y el Teorema~\ref{thm:paper-positivity} implican que
no puede existir un cero fuera de la recta crítica.
\end{proof}
Sea $\widehat f(s)=e^{-(s-\rho_0)^2/\varepsilon}$ suavizada con corte compacto.
Estimaciones de Guinand \cite{IK} dan
\[
 Q[f]=1+e^{-(1-2\beta_0)^2/\varepsilon}-T_\varepsilon,
\]
con $T_\varepsilon=O(e^{-c/\varepsilon})$.  
Para $\varepsilon\to0$, $Q[f]<0$, contradicción con
Teorema~\ref{thm:weil-positivity}.
\end{proof}

\begin{corollary}[Recta crítica]
De los Teoremas \ref{thm:de-branges-selfadjoint}, \ref{thm:weil-positivity} y
Lema \ref{lem:no-off-axis} se deduce que todos los ceros de $D(s)$ están en la
recta crítica.  
\end{corollary}
