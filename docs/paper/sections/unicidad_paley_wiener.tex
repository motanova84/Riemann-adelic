\section{Unicidad Paley--Wiener con multiplicidades}

Establecemos que las propiedades analíticas básicas (orden, simetría, divisor
de ceros y normalización) determinan $D(s)$ de forma única.

\paragraph{Estado actual.}
El argumento requiere controlar rigurosamente el crecimiento de $F$ mediante
Hadamard y Phragm\'en--Lindel\"of; los detalles aún no están documentados y forman
parte del entregable P1.4.

\begin{lemma}[Unicidad]\label{lem:paper-uniqueness}
Sea $F$ una función entera de orden $\leqslant 1$ y tipo finito que satisface
$F(s)=F(1-s)$.  Si el divisor de ceros de $F$ coincide con el de $\Xi(s)$ e
$F(1/2)=\Xi(1/2)$, entonces $F\equiv \Xi$.
\end{lemma}

\begin{proof}
Por la factorización de Hadamard
\cite[Chap.~II]{Tate1967}, el cociente $H(s)=F(s)/\Xi(s)$ es una función entera
sin ceros.  La simetría implica $H(s)=H(1-s)$, de modo que $h(s)=\log H(s)$ es
entera con crecimiento lineal controlado.  El teorema de
Paley--Wiener--Hamburger
\cite[Thm.~5]{Hamburger1921}
identifica $h$ como transformada de Fourier de una medida compactamente
soportada.  La normalización $H(1/2)=1$ obliga a que la medida tenga masa total
nula; si fuese no trivial, $h$ crecería linealmente en alguna dirección
imaginaria, contradiciendo el crecimiento de orden $\leqslant1$.  Por tanto,
$h\equiv0$ y $F=\Xi$.
\end{proof}

Este lema excluye soluciones ``exóticas'': cualquier función entera con las
propiedades postuladas coincide con la función de Riemann completada.
\begin{theorem}[Unicidad con multiplicidades]
Sea $F(s)$ una función entera de orden $\le 1$ y tipo finito, con simetría $F(1-s)=F(s)$.
Suponga que $F$ y $\Xi(s)$ (la función completada de Riemann) tienen la misma medida
espectral de ceros incluyendo multiplicidades y que $F(1/2)=\Xi(1/2)\neq 0$.
Entonces $F\equiv \Xi$.
\end{theorem}

\begin{proof}
Por teoría de Hadamard para funciones enteras de orden $\le 1$, $F$ y $\Xi$
admiten productos canónicos
\[
F(s)=e^{a+bs}\prod_\rho E_1\!\left(\frac{s}{\rho}\right),\qquad
\Xi(s)=e^{a'+b's}\prod_\rho E_1\!\left(\frac{s}{\rho}\right),
\]
donde el producto es sobre los mismos ceros (con multiplicidad) por hipótesis,
y $E_1(z)=(1-z)e^{z}$.
Por tanto, la razón $H(s):=\frac{F(s)}{\Xi(s)}$ es entera sin ceros (y sin polos), luego $H(s)=e^{c+ds}$.

La simetría $F(1-s)=F(s)$ y $\Xi(1-s)=\Xi(s)$ implican
$H(1-s)=H(s)$, es decir $e^{c+d(1-s)}=e^{c+ds}$ para todo $s$, lo que fuerza $d=0$.
Así $H$ es constante. La normalización $F(1/2)=\Xi(1/2)$ fija $H\equiv 1$.
\end{proof}

\begin{lemma}[Control de crecimiento]
Si $F$ y $\Xi$ son de orden $\le 1$, la razón $H$ tiene crecimiento subexponencial en bandas verticales; combinado con la simetría implica $d=0$ incluso sin evaluar en $s=1/2$, siempre que se fije una normalización alternativa (p.ej. el coeficiente principal).
\end{lemma}
