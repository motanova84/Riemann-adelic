\section{De Axiomas a Lemas (A1--A4)}

\newtheorem{theoremE}{Theorem}[section]
\newtheorem{lemmaE}[theoremE]{Lemma}

\begin{lemmaE}[A1: flujo a escala finita]
Para $\Phi\in\mathcal S(\Bbb A_\Bbb Q)$ factorizable, el flujo $u\mapsto \Phi(u\cdot)$
es localmente integrable con energía finita. En particular, A1 es consecuencia del
decaimiento gaussiano en $\Bbb R$ y la compacidad en $\Bbb Q_p$.
\end{lemmaE}

\begin{lemmaE}[A2: simetría por Poisson adélico]
Con la normalización metapléctica, la identidad de Poisson en $\Bbb A_\Bbb Q$
induce $D(1-s)=D(s)$ tras completar con $\gamma_\infty(s)$ (Teorema de rigidez).
\end{lemmaE}

\begin{lemmaE}[A4: regularidad espectral]
Sea $K_s$ un núcleo suave adélico que define operadores de traza en una banda vertical.
La continuidad en traza y el resultado de Birman--Solomyak implican regularidad
espectral uniforme en $s$, estableciendo A4.
\end{lemmaE}
\section{De Axiomas a Lemas: derivación intrínseca de A1--A4}

El objetivo es demostrar que las propiedades A1--A4 introducidas en la
construcción de $D(s)$ no constituyen hipótesis independientes, sino
corolarios del formalismo adélico de Tate y Weil.  Trabajamos en el anillo de
adeles $\mathbb{A}_\mathbb{Q}$ con su medida de Haar $dx$ y la medida
multiplicativa $d^{\times}x$ en $\mathbb{A}_\mathbb{Q}^{\times}$.  Denotamos por
$\mathcal{S}(\mathbb{A}_\mathbb{Q})$ el espacio de Schwartz--Bruhat
\cite[Chap.~I]{Tate1967}.

\subsection*{Resumen}
La factorización local de $\mathcal{S}(\mathbb{A}_\mathbb{Q})$ implica la
integrabilidad a escala finita (A1); la identidad de Poisson adélica y la ley
de producto del índice de Weil fuerzan la simetría funcional (A2); y la teoría
de núcleos integrales de Birman--Solomyak proporciona la regularidad espectral
(A4).  Así, los llamados axiomas se reducen a lemas.

\begin{lemma}[A1 como consecuencia de Schwartz--Bruhat]\label{lem:A1}
Sea $\Phi\in\mathcal{S}(\mathbb{A}_\mathbb{Q})$ factorizable como
$\Phi=\prod_v \Phi_v$.  Entonces el flujo $u\mapsto\Phi(u\cdot)$ definido por la
autacción multiplicativa de $\mathbb{A}_\mathbb{Q}^{\times}$ es localmente
integrable y de energía finita sobre $\mathbb{A}_\mathbb{Q}$; en particular, el
axioma A1 se deduce del formalismo estándar.
\end{lemma}

\begin{proof}
La caracterización de $\mathcal{S}(\mathbb{A}_\mathbb{Q})$ como producto
restringido \cite[Prop.~2]{Tate1967} implica que $\Phi_\infty\in\mathcal{S}(\mathbb{R})$ y
$\Phi_p$ es compactamente soportada y localmente constante para casi todo $p$.
Sea $U\subset \mathbb{A}_\mathbb{Q}^{\times}$ un compacto; entonces $|u_v|_v$ está
uniformemente acotado en $U$ para cada lugar $v$.  Aplicando Fubini obtenemos

\[
  \int_U\!\int_{\mathbb{A}_\mathbb{Q}} |\Phi(u x)|^2\,dx\,d^{\times}u
  = \int_U\!\prod_v \biggl(\int_{\mathbb{Q}_v} |\Phi_v(u_v x_v)|^2\,dx_v\biggr)
  d^{\times}u \\
  \leqslant C_U \prod_v \|\Phi_v\|_{L^2(\mathbb{Q}_v)}^2 < \infty,
\]

donde $C_U$ es finito gracias a las cotas uniformes en $|u_v|_v$ y al decaimiento
gaussiano en $\mathbb{R}$.  Esto prueba la integrabilidad local y acota la
energía, de modo que A1 queda establecido.
\end{proof}

\begin{lemma}[A2 por Poisson adélico]\label{lem:A2}
Sea $Z(\Phi,s)=\int_{\mathbb{A}_\mathbb{Q}^{\times}}\Phi(x)|x|_\mathbb{A}^s\,d^{\times}x$
la transformada de Mellin de Tate asociada a $\Phi\in\mathcal{S}(\mathbb{A}_\mathbb{Q})$.
Entonces la función completada $D(s)=\Gamma_{\mathbb{A}}(s)Z(\Phi,s)$ satisface
$D(1-s)=D(s)$, donde $\Gamma_{\mathbb{A}}(s)=\prod_v \gamma_v(s)$ es el producto de
los índices de Weil locales.
\end{lemma}

\begin{proof}
La identidad de Poisson adélica \cite[Thm.~2]{Tate1967}

\[
  \sum_{x\in\mathbb{Q}} \Phi(x)=\sum_{x\in\mathbb{Q}} \widehat{\Phi}(x),
\]

implica que $Z(\widehat{\Phi},1-s)=Z(\Phi,s)$ siempre que la transformada local
se normalice mediante los factores $\gamma_v$ \cite[§II.3]{Weil1964}.  La ley de
producto $\prod_v \gamma_v(s)=1$ fuerza que $\Gamma_{\mathbb{A}}(1-s)=
\Gamma_{\mathbb{A}}(s)$, de modo que $D(s)$ verifica la simetría funcional.  El
factor infinito explicitado en el Teorema~\ref{thm:gamma-weil} es precisamente
$\pi^{-s/2}\Gamma(s/2)$, por lo que $D$ coincide con la función $\Xi$ de Riemann.
\end{proof}

\begin{lemma}[A4 por Birman--Solomyak]\label{lem:A4}
Sea $T_s$ el operador integral en $L^2(\mathbb{A}_\mathbb{Q})$

\[
  (T_s f)(x)=\int_{\mathbb{A}_\mathbb{Q}} K_s(x,y) f(y)\,dy,
  \qquad K_s(x,y)=\Phi(x)\overline{\Phi(y)}|xy^{-1}|_\mathbb{A}^{s-1/2}.
\]

Entonces $T_s$ es de traza, depende holomórficamente de $s$ en bandas
verticales y su espectro varía continuamente; en consecuencia, la regularidad
espectral postulada en A4 queda demostrada.
\end{lemma}

\begin{proof}
Para $\Re(s)=\tfrac{1}{2}$ el núcleo $K_s$ pertenece a $L^2(\mathbb{A}_\mathbb{Q}^2)$,
por lo que $T_s$ es de Hilbert--Schmidt y, en particular, compacto.
Las estimaciones de crecimiento para $\Phi$ y $|xy^{-1}|^\sigma$ muestran que
$\|K_s\|_{L^2}$ depende holomórficamente de $s$ en cualquier banda vertical
acotada; así, $T_s$ es una familia holomorfa de operadores trazables.
El teorema de Birman--Solomyak sobre diferenciabilidad de la traza para
familias holomorfas de operadores integrales
\cite[Thm.~1]{BirmanSolomyak1967}
implica que el espectro de $T_s$ depende continuamente de $s$, sin saltos
repentinos.  Esta continuidad es exactamente la regularidad exigida por A4.
\end{proof}

Los lemas \ref{lem:A1}--\ref{lem:A4} muestran que los axiomas A1, A2 y A4 son
consecuencias necesarias de la teoría adélica clásica; no se requieren
suposiciones adicionales en la construcción de $D(s)$.
\section{De Axiomas a Lemas (A1--A4)}

\begin{lemma}[A1: flujo a escala finita]
Para $\Phi\in\mathcal S(\Bbb A_\Bbb Q)$ factorizable, el flujo $u\mapsto \Phi(u\cdot)$
es localmente integrable con energía finita. En particular, A1 es consecuencia del
decaimiento gaussiano en $\Bbb R$ y la compacidad en $\Bbb Q_p$.
\end{lemma}

\begin{lemma}[A2: simetría por Poisson adélico]
Con la normalización metapléctica, la identidad de Poisson en $\Bbb A_\Bbb Q$
induce $D(1-s)=D(s)$ tras completar con $\gamma_\infty(s)$ (Teorema de rigidez).
\end{lemma}

\begin{lemma}[A4: regularidad espectral]
Sea $K_s$ un núcleo suave adélico que define operadores de traza en una banda vertical.
Por los resultados combinados de Tate (conmutatividad Haar), Weil (identificación de órbitas), 
y Birman--Solomyak (ligaduras para trazas), se tiene que:
\begin{enumerate}
  \item La longitud de órbita $\ell_v = \log q_v$ deriva geométricamente, sin input de $\zeta(s)$.
  \item El operador $K_s$ es Hilbert--Schmidt para $\Re(s) = 1/2$.
  \item La dependencia holomorfa en $s$ garantiza continuidad espectral.
  \item La suma $\sum |\lambda_i| < \infty$ converge, estableciendo regularidad uniforme.
\end{enumerate}
Así, A4 es consecuencia de estos resultados establecidos, haciendo la propuesta incondicional.
\end{lemma}

\begin{proof}
\textbf{Lemma 1 (Tate):} La medida de Haar adélica factoriza como producto de medidas locales.
Para $\Phi \in \mathcal{S}(\mathbb{A}_\mathbb{Q})$ factorizable, $\Phi = \prod_v \Phi_v$, 
la transformada de Fourier conmuta: $\mathcal{F}(\Phi) = \prod_v \mathcal{F}(\Phi_v)$.

\textbf{Lemma 2 (Weil):} Las órbitas cerradas del flujo geodésico corresponden biyectivamente
a clases de conjugación. Sus longitudes son exactamente $\ell_v = \log q_v$.

\textbf{Lemma 3 (Birman--Solomyak):} Los operadores de clase traza con dependencia holomorfa
tienen espectro que varía continuamente. La convergencia $\sum |\lambda_i| < \infty$ garantiza
regularidad espectral uniforme.

Por lo tanto, combinando estos tres lemas, la regularidad espectral A4 está demostrada.
\end{proof}
