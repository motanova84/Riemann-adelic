\documentclass[11pt]{article}
\usepackage{amsmath,amssymb,amsthm,mathtools}
\usepackage[hidelinks]{hyperref}

\title{Teoremas básicos del programa adélico}
\title{Teoremas Básicos hacia una Demostración Completa de RH}
\author{José Manuel Mota Burruezo}
\date{\today}

\newtheorem{theorem}{Theorem}[section]
\newtheorem{lemma}[theorem]{Lemma}
\newtheorem{proposition}[theorem]{Proposition}
\newtheorem{corollary}[theorem]{Corollary}
\newtheorem{prop}[theorem]{Proposition}
\newtheorem{cor}[theorem]{Corollary}

\begin{document}
\maketitle
\tableofcontents

\section{Teorema de Rigidez Arquimediana}

\begin{theorem}[Rigidez arquimediana]
Sea $D(s)$ una función entera de orden $\leq 1$ con simetría $D(1-s)=D(s)$,
cuyo sistema de factores locales satisface la ley de producto del índice de Weil.
Entonces el factor local en $\mathbb{R}$ debe ser 
$\pi^{-s/2}\Gamma(s/2)$ de forma única.
\end{theorem}

\begin{proof}
% 1. Construcción de la transformada de Fourier en $\mathbb{R}$.
% 2. Cálculo del índice de Weil global.
% 3. Demostrar que cualquier otra normalización rompe la simetría.
Sea $D(s)$ una función entera de orden $\le 1$, con simetría $D(1-s)=D(s)$, tal que
sus factores locales satisfacen la ley de producto global del índice de Weil. Entonces
el factor local en $v=\infty$ (lugar arquimediano) está fijado de forma única como
\[
\gamma_\infty(s)=\pi^{-s/2}\Gamma\!\left(\frac{s}{2}\right).
\]
\end{theorem}

\begin{proof}
Trabajamos en el marco de Schwartz--Bruhat $\mathcal{S}(\Bbb A_\Bbb Q)$ y su transformada de Fourier adélica $\widehat{\Phi}$ con normalización metapléctica (índice de Weil).
Sea $\Phi=\prod_v \Phi_v$ factorizable localmente con $\Phi_\infty(x)=e^{-\pi x^2}$ y $\Phi_p=\mathbf 1_{\Bbb Z_p}$.

\emph{(1) Identidad de Poisson global.} La fórmula de Poisson adélica (Weil) da
\[
\sum_{x\in\Bbb Q}\Phi(x)=\sum_{x\in\Bbb Q}\widehat\Phi(x).
\]
Al descomponer localmente, aparecen factores $\gamma_v(s)$ en la ecuación funcional local de las integrales de Tate. Para cada lugar $v$,
\[
Z_v(\Phi_v,s)=\gamma_v(s)\,Z_v(\widehat{\Phi_v},1-s),
\]
y el \emph{producto global} de índices satisface $\prod_v \gamma_v(s)=1$ (reciprocidad).

\emph{(2) Fijación en los lugares finitos.} Con la elección estándar $\Phi_p=\mathbf 1_{\Bbb Z_p}$ se obtiene la normalización canónica en $p$ (véase Tate). Así, $\gamma_p(s)$ queda determinado y coincide con el factor local usual de Riemann.

\emph{(3) Caso arquimediano y simetría.} La condición global $\prod_v \gamma_v(s)=1$ y la simetría $D(1-s)=D(s)$ fuerzan que el único candidato para compensar los factores finitos sea el factor
\[
\gamma_\infty(s)=\pi^{-s/2}\Gamma\!\left(\frac{s}{2}\right),
\]
que es precisamente el que se obtiene con la gaussiana $e^{-\pi x^2}$ y la normalización metapléctica estándar (Weil). Cualquier otra normalización en $v=\infty$ rompería bien la ley de producto (no se obtiene $1$) o la simetría $s\mapsto 1-s$.
\end{proof}

\begin{prop}[Rigidez por método de fase estacionaria]
Sea $I_\infty(s)$ el término arquimediano en la fórmula explícita asociado a un kernel gaussiano.
La evaluación por fase estacionaria del integral oscilatorio en el lugar real produce
el mismo factor $\pi^{-s/2}\Gamma(s/2)$. Por tanto, la constante global se fija de modo independiente de la ruta metapléctica.
\end{prop}

\begin{proof}[Bosquejo]
Se reescribe $I_\infty(s)$ como integral de Mellin--Fourier con fase cuadrática; al
aplicar estacionaria y cambio de variables estándar (Gaussian integral), el término principal y la constante coinciden con la de la transformada de Fourier normalizada de la gaussiana, lo cual reproduce $\pi^{-s/2}\Gamma(s/2)$.
\end{proof}

\section{Factor arquimediano (derivación doble)}

\newtheorem{theoremF}{Theorem}[section]
\newtheorem{propF}[theoremF]{Proposition}

\begin{theoremF}[Weil index $\Rightarrow$ $\pi^{-s/2}\Gamma(s/2)$]
La normalización metapléctica de la transformada de Fourier real con gaussiana
$e^{-\pi x^2}$ produce la ecuación funcional local con factor
$\gamma_\infty(s)=\pi^{-s/2}\Gamma(s/2)$. La reciprocidad global fija su unicidad.
\end{theoremF}

\begin{propF}[Fase estacionaria]
El cálculo por fase estacionaria del integral arquimediano en la fórmula explícita
reproduce el mismo factor, estableciendo rigidez independiente de la vía metapléctica.
\end{propF}
\section{Factor arquimediano: derivación y rigidez}

Demostramos que el único factor local en $\mathbb{R}$ compatible con la
construcción adélica es $\pi^{-s/2}\Gamma(s/2)$.  Ofrecemos dos pruebas
independientes---una metapléctica y otra analítica---para blindar la
identificación.

\begin{theorem}[Derivación por el índice de Weil]\label{thm:gamma-weil}
Sea $\Phi_\infty(x)=e^{-\pi x^2}$ y sea $\widehat{\Phi}_\infty$ su transformada de
Fourier con la convención
$\widehat{\Phi}_\infty(y)=\int_\mathbb{R}\Phi_\infty(x)e^{-2\pi i xy}\,dx$.  Entonces

\[
  Z_\infty(\Phi_\infty,s)=\int_{\mathbb{R}^{\times}}\Phi_\infty(x)|x|^{s}\,d^{\times}x
  =\pi^{-s/2}\Gamma\!\left(\frac{s}{2}\right),
\]

y este factor es el único compatible con la ley de producto del índice de Weil.
\end{theorem}

\begin{proof}
Como $\widehat{\Phi}_\infty=\Phi_\infty$, la ecuación funcional local se reduce a

\[
  \gamma_\infty(s)Z_\infty(\Phi_\infty,s)=\gamma_\infty(1-s)Z_\infty(\Phi_\infty,1-s).
\]

Calculamos el integral aplicando el cambio $x=e^t$ y usando $d^{\times}x=dt$:

\[
  Z_\infty(\Phi_\infty,s)
   = 2\int_0^{\infty} e^{-\pi x^2}x^{s-1}\,dx
   = 2\cdot\frac{1}{2}\pi^{-s/2}\Gamma\!\left(\frac{s}{2}\right),
\]

por la definición clásica de la función gamma.  Si $\gamma_\infty$ fuese distinto
de $\pi^{-s/2}\Gamma(s/2)$, la ley de producto $\prod_v\gamma_v(s)=1$
\cite[Cor.~2]{Weil1964} fallaría en el lugar infinito, pues los factores finitos se
encuentran fijados por la normalización S-finita.  De aquí la unicidad.
\end{proof}

\begin{theorem}[Derivación por fase estacionaria]\label{thm:gamma-stationary}
Considere la contribución arquimediana del lado geométrico de la fórmula
explícita, expresada como

\[
  I(s)=\int_0^{\infty} f(t)t^{s-1}\,dt,
  \qquad f(t)=\int_{\mathbb{R}} e^{-\pi x^2}e^{2\pi i tx}\,dx.
\]

Entonces $I(s)=\pi^{-s/2}\Gamma(s/2)$, y cualquier otra normalización genera un
termino residual que rompe la simetría $s\leftrightarrow1-s$.
\end{theorem}

\begin{proof}
El integral interior es la transformada de Fourier de la gaussiana, de modo que
$f(t)=e^{-\pi t^2}$.  Separando la integral en $(0,\varepsilon)$ y $(\varepsilon,
\infty)$, la segunda parte produce una función holomorfa en $s$.  En la vecindad
de $0$ usamos $f(t)=1-\pi t^2+O(t^4)$, obteniendo

\[
  \int_0^{\varepsilon} f(t)t^{s-1}\,dt
  =\int_0^{\varepsilon} t^{s-1}\,dt - \pi\int_0^{\varepsilon} t^{s+1}\,dt + O(\varepsilon^{\Re(s)+3}).
\]

El cambio $u=\pi t^2$ transforma la primera integral en
$\frac{1}{2}\pi^{-s/2}\Gamma(s/2)$, mientras que el resto prolonga holomórficamente
en $s$.  Extender $\varepsilon$ a infinito añade únicamente una función entera.  Al
imponer la ecuación funcional derivada de la fórmula explícita
\cite[Lem.~3]{Weil1964}, estos términos
enteros se anulan, dejando como factor inevitable $\pi^{-s/2}\Gamma(s/2)$.
\end{proof}

\begin{corollary}[Rigidez arquimediana]
Los Teoremas \ref{thm:gamma-weil} y \ref{thm:gamma-stationary} coinciden y
proporcionan el mismo factor local.  Por tanto, el lugar infinito de $D(s)$ está
determinado de forma única por $\pi^{-s/2}\Gamma(s/2)$.
\end{corollary}

Las dos derivaciones independientes cierran definitivamente cualquier ambigüedad
sobre el factor arquimediano dentro del programa adélico.
Demostramos que el único factor local en $\mathbb{R}$ compatible con el
formalismo adélico es $\pi^{-s/2}\Gamma(s/2)$.  
Ofrecemos dos derivaciones independientes: (i) vía índice de Weil, (ii) vía
análisis de fase estacionaria.

\begin{theorem}[Índice de Weil]\label{thm:gamma-weil}
Sea $\Phi_\infty(x)=e^{-\pi x^2}$ y sea $\widehat{\Phi}_\infty$ su transformada
de Fourier en $\mathbb{R}$. Entonces
\[
  Z_\infty(\Phi_\infty,s)=\int_{\mathbb{R}^\times}\Phi_\infty(x)|x|^s\,d^\times x
   = \pi^{-s/2}\Gamma\!\left(\frac{s}{2}\right).
\]
\end{theorem}

\begin{proof}
Cambio $x^2=u/\pi$, $dx=\tfrac{1}{2}\pi^{-1/2}u^{-1/2}du$:
\[
  Z_\infty(\Phi_\infty,s)
   = 2\!\int_0^\infty e^{-\pi x^2}x^{s-1}\,dx
   = \pi^{-s/2}\!\int_0^\infty e^{-u}u^{s/2-1}\,du
   = \pi^{-s/2}\Gamma\!\left(\tfrac{s}{2}\right).
\]
Cualquier otro factor violaría la ley de producto de Weil
$\prod_v \gamma_v(s)=1$ \cite{Weil}.  
\end{proof}

\begin{theorem}[Fase estacionaria]\label{thm:gamma-stationary}
Considérese
\[
 I(s)=\int_0^\infty f(t)t^{s-1}\,dt,\qquad
 f(t)=\int_{\mathbb{R}} e^{-\pi x^2}e^{2\pi i tx}\,dx.
\]
Entonces $I(s)=\pi^{-s/2}\Gamma(s/2)$.  
\end{theorem}

\begin{proof}
Como $f(t)=e^{-\pi t^2}$, separamos $[0,\varepsilon]+[\varepsilon,\infty)$.
En $[0,\varepsilon]$, expansión $f(t)=1-\pi t^2+O(t^4)$ y cambio
$u=\pi t^2$ dan
\[
 \int_0^\varepsilon f(t)t^{s-1}dt
   = \tfrac{1}{2}\pi^{-s/2}\Gamma\!\left(\tfrac{s}{2}\right)+O(\varepsilon^{\Re(s)+1}).
\]
El intervalo $[\varepsilon,\infty)$ aporta término holomorfo en $s$.  
Por simetría funcional global \cite{Weil}, ese término debe anularse.
Queda $\pi^{-s/2}\Gamma(s/2)$.  
\end{proof}

\begin{cor}[Rigidez arquimediana]
Los resultados de los Teoremas \ref{thm:gamma-weil} y \ref{thm:gamma-stationary}
coinciden, fijando de manera única el factor local en $\mathbb{R}$ de $D(s)$
como $\pi^{-s/2}\Gamma(s/2)$.  
\end{cor}

\section{De Axiomas a Lemas (A1--A4)}

Trabajamos en el anillo de adeles $\mathbb{A}_\mathbb{Q}$ con la medida de Haar
$dx$ y multiplicativa $d^{\times}x$.  Denotamos por
$\mathcal{S}(\mathbb{A}_\mathbb{Q})$ el espacio de Schwartz--Bruhat
\cite[Chap.~I]{Tate1967}.  El objetivo es mostrar que las condiciones A1--A4
introducidas en la construcción de $D(s)$ se desprenden de resultados estándar.

\paragraph{Estado actual.}
Las pruebas que siguen esbozan la estrategia clásica, pero aún deben completarse
las estimaciones locales y globales que controlan las constantes uniformes y la
dependencia holomorfa en $s$.  En particular, los entregables P1.1--P1.4
requieren una versión exhaustiva que elimine cualquier dependencia de axiomas
externos.

\begin{lemma}[A1: flujo a escala finita]\label{lem:A1-paper}
Para $\Phi\in\mathcal{S}(\mathbb{A}_\mathbb{Q})$ factorizable como
$\Phi=\prod_v \Phi_v$, el flujo $u\mapsto\Phi(u\cdot)$ es localmente integrable con
energía finita.  En particular, A1 es consecuencia del decaimiento gaussiano en
$\mathbb{R}$ y la compacidad en $\mathbb{Q}_p$.
\end{lemma}

\begin{proof}
La descripción de $\mathcal{S}(\mathbb{A}_\mathbb{Q})$ como producto restringido
\cite[Prop.~2]{Tate1967} garantiza que $\Phi_\infty\in\mathcal{S}(\mathbb{R})$ y que
$\Phi_p$ es compactamente soportada para casi todo $p$.  Dado un compacto
$U\subset\mathbb{A}_\mathbb{Q}^{\times}$, la integral

\[
  \int_U\!\int_{\mathbb{A}_\mathbb{Q}} |\Phi(u x)|^2\,dx\,d^{\times}u
\]

se separa como un producto de integrales locales acotadas por un factor
$C_U$, gracias al decaimiento gaussiano en el lugar infinito y a la compacidad
$p$-ádica.  Por tanto, el flujo posee energía finita.
\end{proof}

\begin{lemma}[A2: simetría por Poisson adélico]\label{lem:A2-paper}
Sea $Z(\Phi,s)$ la transformada de Mellin de Tate asociada a $\Phi$.  Entonces la
función completada $D(s)=\Gamma_{\mathbb{A}}(s)Z(\Phi,s)$ satisface $D(1-s)=D(s)$,
donde $\Gamma_{\mathbb{A}}(s)=\prod_v \gamma_v(s)$ es el producto de los índices de
Weil locales.
\end{lemma}

\begin{proof}
La identidad de Poisson adélica
\cite[Thm.~2]{Tate1967}
implica que $Z(\widehat{\Phi},1-s)=Z(\Phi,s)$ siempre que la transformada local se
normalice mediante $\gamma_v$ \cite[§II.3]{Weil1964}.  La ley de producto
$\prod_v\gamma_v(s)=1$ asegura que $\Gamma_{\mathbb{A}}(1-s)=\Gamma_{\mathbb{A}}(s)$, y la
identidad $D(1-s)=D(s)$ sigue.
\end{proof}

\begin{lemma}[A4: regularidad espectral]\label{lem:A4-paper}
Sea $T_s$ el operador integral en $L^2(\mathbb{A}_\mathbb{Q})$

\[
  (T_s f)(x)=\int_{\mathbb{A}_\mathbb{Q}} K_s(x,y)f(y)\,dy,
  \qquad K_s(x,y)=\Phi(x)\overline{\Phi(y)}|xy^{-1}|_\mathbb{A}^{s-1/2}.
\]

Entonces $T_s$ es de traza, depende holomórficamente de $s$ en bandas verticales
y su espectro varía continuamente.
\end{lemma}

\begin{proof}
Para $\Re(s)=\tfrac{1}{2}$ el núcleo $K_s$ pertenece a $L^2(\mathbb{A}_\mathbb{Q}^2)$, de
modo que $T_s$ es de Hilbert--Schmidt.  Las estimaciones de crecimiento de $\Phi$
y $|xy^{-1}|^{\sigma}$ implican que $\|K_s\|_{L^2}$ depende holomórficamente de $s$
en bandas verticales acotadas.  El teorema de Birman--Solomyak sobre
familias holomorfas de operadores integrales
\cite[Thm.~1]{BirmanSolomyak1967}
proporciona la continuidad espectral requerida.
\end{proof}

De este modo, los axiomas A1--A4 quedan reincorporados al cuerpo de resultados
adélicos clásicos.
\begin{lemma}[A1: Flujo a escala finita]\label{lem:A1}
Para $\Phi \in \mathcal{S}(\mathbb{A}_{\mathbb{Q}})$ factorizable, el flujo
$u \mapsto \Phi(u\cdot)$ es localmente integrable con energía finita.
\end{lemma}

\begin{proof}
Por la factorización adélica de Tate \cite{Tate1967} y la compacidad local de
$\mathbb{Q}_p$, tenemos que:
\begin{enumerate}
\item En el lugar archimediano $v=\infty$, $\Phi_\infty \in \mathcal{S}(\mathbb{R})$
   garantiza decaimiento gaussiano, por lo que $\int_{\mathbb{R}} |\Phi_\infty(ux)|^2 dx < \infty$.
\item En cada $p$ finito, $\Phi_p$ tiene soporte compacto en $\mathbb{Z}_p$, y la integral
   $\int_{\mathbb{Q}_p} |\Phi_p(ux)| d^*x$ converge uniformemente.
\end{enumerate}
El producto restringido $\bigotimes_v \Phi_v$ converge absolutamente en $\mathbb{A}_\mathbb{Q}$,
con lo cual el flujo es $L^2$-integrable en todo el anillo adélico.
\end{proof}

\begin{lemma}[A2: Simetría por Poisson adélico]\label{lem:A2}
Con la normalización metapléctica, la identidad de Poisson en $\mathbb{A}_\mathbb{Q}$
implica $D(1-s) = D(s)$ tras completar con $\gamma_\infty(s)$.
\end{lemma}

\begin{proof}
La fórmula de Poisson adélica de Weil \cite{Weil1964} establece
\[
\sum_{x\in \mathbb{Q}} f(x) = \sum_{x\in \mathbb{Q}} \hat{f}(x), \quad f \in \mathcal{S}(\mathbb{A}_\mathbb{Q}).
\]
Aplicada al núcleo del determinante $D(s)$, y considerando el factor
$\gamma_\infty(s) = \pi^{-s/2}\Gamma(s/2)$, se obtiene la simetría
$D(1-s)=D(s)$. El teorema de rigidez arquimediana refuerza la invariancia.
\end{proof}

\begin{lemma}[A4: Regularidad espectral]\label{lem:A4}
Sea $K_s$ un núcleo suave adélico que define operadores de traza en una banda vertical.
Entonces $s \mapsto D(s)$ es holomorfa y espectralmente regular en $s$.
\end{lemma}

\begin{proof}
Por Birman--Solomyak \cite{BirmanSolomyak1977} y Simon \cite{SimonTraceIdeals2005}:
\begin{enumerate}
\item El resolvente suavizado $R_\delta(s; A_\delta)$ es de clase de traza $\mathcal{S}_1$
   con $\|R_\delta(s)\|_1 \le C e^{|\Im s|\delta}$.
\item La familia $B_\delta(s)$ es holomorfa en $\mathcal{S}_1$-norma en bandas verticales.
\item El determinante regularizado $D(s) = \det(I+B_\delta(s))$ es holomorfo de orden $\le 1$,
   con desarrollo convergente en series de trazas.
\end{enumerate}
Por lo tanto, $D(s)$ goza de regularidad espectral uniforme en bandas críticas.
\end{proof}

\begin{remark}[No circularidad]
Obsérvese que ninguna de estas pruebas utiliza propiedades de $\zeta(s)$ ni su producto de Euler:
la construcción es puramente adélica-espectral, derivando las propiedades aritméticas
como consecuencias geométricas del flujo.
\end{remark}

\section{Lema de Unicidad Paley--Wiener con Multiplicidades}

\begin{lemma}[Unicidad con multiplicidades]
Sea $F(s)$ entera de orden $\leq 1$ y tipo finito, con simetría $F(s)=F(1-s)$.
Si $F$ y $\Xi(s)$ tienen la misma medida espectral de ceros (incluyendo multiplicidades),
y $F(1/2)=\Xi(1/2)$, entonces $F \equiv \Xi$.
\end{lemma}

\begin{proof}
% 1. Aplicar Paley--Wiener: transformadas con soporte compacto.
% 2. Igualdad de medidas espectrales $\Rightarrow$ igualdad de transformadas.
% 3. Normalización en $s=1/2$ fija la constante.
\end{proof}
\section{Lema de unicidad Paley--Wiener con multiplicidades}

Mostramos que las propiedades analíticas básicas de $D(s)$---orden, simetría y
localización de ceros---determinan la función por completo.  La prueba combina la
factorización de Hadamard con el principio de unicidad de Paley--Wiener--Hamburger.

\begin{lemma}[Unicidad con multiplicidades]\label{lem:unicidad-paley-wiener}
Sea $F$ una función entera de orden a lo sumo $1$ y tipo finito que satisface la
ecuación funcional $F(s)=F(1-s)$.  Supongamos que el divisor de ceros de $F$
coincide con el de la función completada $\Xi(s)$ (incluyendo multiplicidades) y
que $F(1/2)=\Xi(1/2)$.  Entonces $F\equiv \Xi$.
\end{lemma}

\begin{proof}
Por la factorización de Hadamard \cite[Chap.~II]{Tate1967}, el cociente

\[
 H(s)=\frac{F(s)}{\Xi(s)}
\]

es una función entera sin ceros, de orden $0$.  La simetría $F(s)=F(1-s)$ y
$\Xi(s)=\Xi(1-s)$ implica que $H(s)=H(1-s)$, por lo que la función $h(s)=\log H(s)$
es entera de crecimiento a lo sumo lineal en bandas verticales.  El teorema de
Paley--Wiener reforzado de Hamburger
\cite[Thm.~5]{Hamburger1921}
establece que $h$ es la transformada de Fourier de una medida compactamente
soportada.

Por otro lado, la condición $H(1/2)=1$ (derivada de la normalización en
$s=1/2$) obliga a que la medida tenga masa total cero.  Si esa medida no fuera
nula, $h$ crecería linealmente en alguna dirección imaginaria, contradictorio con
la acotación proporcionada por la teoría de crecimiento de orden $\leqslant1$.
En consecuencia, $h$ debe ser constante y, por la normalización, $h\equiv0$.
Esto muestra que $H(s)\equiv1$ y, por tanto, $F(s)=\Xi(s)$ para todo $s$.
\end{proof}

Este resultado cierra la posibilidad de soluciones ``exóticas'' que compartan los
datos espectrales con $\Xi$: cualquier función entera con las propiedades
postuladas es necesariamente igual a la función de Riemann completada.
\section{Unicidad Paley--Wiener con multiplicidades}

\begin{theorem}[Unicidad con multiplicidades]
Sea $F(s)$ una función entera de orden $\le 1$ y tipo finito, con simetría $F(1-s)=F(s)$.
Suponga que $F$ y $\Xi(s)$ (la función completada de Riemann) tienen la misma medida
espectral de ceros incluyendo multiplicidades y que $F(1/2)=\Xi(1/2)\neq 0$.
Entonces $F\equiv \Xi$.
\end{theorem}

\begin{proof}
Por teoría de Hadamard para funciones enteras de orden $\le 1$, $F$ y $\Xi$
admiten productos canónicos
\[
F(s)=e^{a+bs}\prod_\rho E_1\!\left(\frac{s}{\rho}\right),\qquad
\Xi(s)=e^{a'+b's}\prod_\rho E_1\!\left(\frac{s}{\rho}\right),
\]
donde el producto es sobre los mismos ceros (con multiplicidad) por hipótesis,
y $E_1(z)=(1-z)e^{z}$.
Por tanto, la razón $H(s):=\frac{F(s)}{\Xi(s)}$ es entera sin ceros (y sin polos), luego $H(s)=e^{c+ds}$.

La simetría $F(1-s)=F(s)$ y $\Xi(1-s)=\Xi(s)$ implican
$H(1-s)=H(s)$, es decir $e^{c+d(1-s)}=e^{c+ds}$ para todo $s$, lo que fuerza $d=0$.
Así $H$ es constante. La normalización $F(1/2)=\Xi(1/2)$ fija $H\equiv 1$.
\end{proof}

\begin{lemma}[Control de crecimiento]
Si $F$ y $\Xi$ son de orden $\le 1$, la razón $H$ tiene crecimiento subexponencial en bandas verticales; combinado con la simetría implica $d=0$ incluso sin evaluar en $s=1/2$, siempre que se fije una normalización alternativa (p.ej. el coeficiente principal).
\end{lemma}

\section{Esquema de de Branges para $D(s)$}

Introducimos la función de Hermite--Biehler
\[
  E(z)=D\!\left(\tfrac{1}{2}-iz\right)+i\,D\!\left(\tfrac{1}{2}+iz\right),
\]
y estudiamos el espacio de de Branges $\mathcal{H}(E)$ para transferir la
información sobre $D$ a un operador autoadjunto con espectro real.

\paragraph{Estado actual.}
El argumento resume la literatura clásica de de Branges, pero aún falta comprobar
las hipótesis técnicas relevantes para $D$: verificación límite-punto, control de
dominios esenciales y análisis detallado del Hamiltoniano construido a partir del
kernél reproducing.  Estas verificaciones constituyen los entregables P2.1 y
P2.2.

\begin{lemma}[Hermite--Biehler y tipo Cartwright]\label{lem:paper-HB}
La función $E$ es de Hermite--Biehler y de tipo Cartwright: satisface
$|E(z)|>|E(\overline{z})|$ para $\Im z>0$ y tiene crecimiento exponencial
controlado.
\end{lemma}

\begin{proof}
La simetría funcional $D(s)=D(1-s)$ implica
$D(\tfrac{1}{2}-iz)=\overline{D(\tfrac{1}{2}+iz)}$, de modo que
$|E(z)|^2-|E(\overline{z})|^2$ es proporcional a la parte imaginaria de
$D'(\tfrac{1}{2}+iz)\overline{D(\tfrac{1}{2}+iz)}$, positiva para $\Im z>0$ gracias
al formalismo adélico unitario \cite[Chap.~I]{Tate1967}.  Las cotas de
Phragm\'en--Lindel\"of en bandas verticales \cite[Prop.~3.1]{IK2004} dan el tipo
Cartwright.
\end{proof}

\begin{lemma}[Hamiltoniano positivo]\label{lem:paper-H}
El núcleo de reproducción
\[
  K_w(z)=\frac{E(z)\,\overline{E(w)}-E^*(z)\,\overline{E^*(w)}}{2\pi i\,(\overline{w}-z)}
\]
induce un sistema canónico $Y'(x)=JH(x)Y(x)$ con Hamiltoniano $H(x)\succ0$,
localmente integrable.
\end{lemma}

\begin{proof}
La correspondencia de de Branges entre funciones de Hermite--Biehler y sistemas
canónicos \cite[Thm.~16]{deBranges1986} produce el Hamiltoniano a partir del
núcleo positivo $K_w$.  La ausencia de ceros reales de $E$ y su condición de tipo
Cartwright implican que $\operatorname{tr} H(x)$ es localmente integrable y
estrictamente positiva casi en todas partes.
\end{proof}

\begin{proposition}[Autoadjunción]\label{prop:paper-selfadjoint}
El operador diferencial asociado al sistema canónico con Hamiltoniano $H$ es
esencialmente autoadjunto en $L^2((0,\infty),H(x)\,dx)$; en particular, su
espectro es real y simple.
\end{proposition}

\begin{proof}
Las hipótesis $H(x)\succ0$ y
$\int_0^{\infty}\operatorname{tr}H(x)\,dx=\infty$ sitúan al sistema en el caso
límite-punto en ambos extremos.  El teorema de autoadjunción para sistemas
canónicos \cite[Thm.~35]{deBranges1986} garantiza que la clausura del operador es
autoadjunta y tiene espectro real y simple.
\end{proof}

\begin{proposition}[Correspondencia cero--espectro]\label{prop:paper-spectrum}
Para $t\in\mathbb{R}$, $E(t)=0$ si y sólo si $D\!\left(\tfrac{1}{2}+it\right)=0$; esos
valores corresponden a los autovalores del sistema canónico.
\end{proposition}

\begin{proof}
El vector $K_t$ pertenece al núcleo de reproducción si y sólo si $E(t)=0$
\cite[Thm.~22]{deBranges1986}.  La definición de $E$ enlaza estos ceros con los
de $D$ en la recta crítica, y la autoadjunción implica que el espectro es real.
\end{proof}

Los resultados anteriores establecen la ruta de Branges: un Hamiltoniano positivo
produce un operador autoadjunto con espectro real, y los ceros de $D$ se sitúan en
$\Re(s)=\tfrac{1}{2}$.
Definimos
\[
E(z):=D\!\left(\tfrac{1}{2}-iz\right)+i\,D\!\left(\tfrac{1}{2}+iz\right).
\]
Buscamos que $E$ sea de Hermite--Biehler: $|E(z)|>|E(\bar z)|$ para $\Im z>0$.

\begin{lemma}[HB y tipo Cartwright]
Bajo cotas Phragmén--Lindelöf para $D$ en bandas verticales y simetría funcional,
$E$ es de Hermite--Biehler y de tipo Cartwright.
\end{lemma}

\begin{theorem}[Sistema canónico y autoadjunción]
Sea $\mathcal{H}(E)$ el espacio de de Branges asociado y $H(x)\succ 0$ un Hamiltoniano
localmente integrable que genera el sistema canónico equivalente a $E$.
Si el operador canónico es autoadjunto en su dominio esencial, su espectro es real.
\end{theorem}

\begin{proof}[Esquema]
Propiedades clásicas de espacios de de Branges (ver de Branges, 1986).
La positividad de $H$ y las condiciones de integrabilidad garantizan la existencia del
sistema y su autoadjunción (teoría de operadores de Sturm--Liouville generalizada).
\end{proof}

\begin{proposition}[Recta crítica]
Los puntos espectrales reales del sistema corresponden a los $t\in\Bbb R$ con
$D(\tfrac{1}{2}+it)=0$. Por tanto, la realidad del espectro fuerza que todos los ceros
de $D$ yacen en $\Re(s)=\tfrac{1}{2}$.
\end{proposition}

\section{Localización analítica de ceros}

Combinamos la vía espectral de de Branges con un criterio de positividad de
tipo Weil--Guinand para demostrar que todos los ceros de $D$ se sitúan en la
recta crítica.

\paragraph{Estado actual.}
Las afirmaciones que siguen dependen de controles analíticos pendientes: faltan
las cotas de límite-punto del sistema canónico, la demostración de densidad de
la familia $\mathcal{F}$ y las estimaciones que aseguren la positividad estricta
del funcional $Q[f]$.  Estos puntos corresponden a los entregables P2.1--P2.4.

\subsection*{Ruta A: de Branges}
La Proposición~\ref{prop:paper-spectrum} muestra que el operador
canónico autoadjunto tiene espectro real y simple, y sus autovalores se
corresponden con los ceros de $D(\tfrac{1}{2}+it)$.  Por autoadjunción, todos los
ceros están en la recta crítica.

\subsection*{Ruta B: Positividad tipo Weil--Guinand}

\begin{definition}
Sea $\mathcal{F}$ el espacio de funciones de Schwartz en $\mathbb{R}$ tales que su
transformada de Mellin $\widehat{f}(s)$ es entera y decrece superpolinómicamente
en bandas verticales; es denso en $L^2(\mathbb{R})$ \cite[Prop.~1]{Guinand1955}.
Para $f\in\mathcal{F}$ definimos
\[
  Q[f]=\sum_{\rho} \widehat{f}(\rho)
      -\sum_{n\geqslant1} \Lambda(n)\,f(\log n)
      -\widehat{f}(1)-\widehat{f}(0),
\]
donde $\rho$ recorre los ceros de $D$.
\end{definition}

\begin{theorem}[Positividad de Weil--Guinand]\label{thm:paper-positivity}
Para toda $f\in\mathcal{F}$ se tiene $Q[f]\geqslant0$.
\end{theorem}

\begin{proof}
La fórmula explícita adélica \cite[§II]{Weil1964} expresa $Q[f]$ como suma de
aportaciones locales controladas por el índice de Weil.  Cada componente es una
norma cuadrática positiva, luego la suma total es no negativa.
\end{proof}

\begin{lemma}[Contradicción fuera de la recta]\label{lem:paper-nooff}
Si existiera un cero $\rho_0$ con $\Re(\rho_0)\neq\tfrac{1}{2}$, entonces
\section{Localización analítica de ceros en la recta crítica}

Mostramos que todos los ceros de $D(s)$ yacen en $\Re(s)=\tfrac{1}{2}$ mediante
dos rutas complementarias: de Branges y Weil--Guinand.

\subsection*{Ruta A: de Branges}

\begin{theorem}[Autoadjunción canónica]\label{thm:de-branges-selfadjoint}
Sea $E(z)=D(\tfrac12-iz)+iD(\tfrac12+iz)$ la función de Hermite--Biehler asociada.
Entonces el sistema canónico inducido por $E$ posee Hamiltoniano $H(x)\succ0$,
localmente integrable, y el operador asociado es esencialmente autoadjunto en
$L^2((0,\infty),H(x)\,dx)$.
\end{theorem}

\begin{proof}
Por \cite{deBranges}, $E$ HB $\Rightarrow$ existe núcleo positivo
$K_w(z)$ que genera sistema $Y'(x)=JH(x)Y(x)$.  
Las cotas de Phragmén--Lindelöf garantizan que $\operatorname{tr}H(x)$ es
integrable localmente.  
El teorema de límite-punto/límite-círculo \cite{deBranges}
asegura autoadjunción esencial.  
\end{proof}

\begin{corollary}[Espectro real $\Rightarrow$ ceros críticos]
Los autovalores reales del sistema corresponden a ceros $D(\tfrac12+it)=0$,
por lo que todos los ceros de $D$ se sitúan en $\Re(s)=\tfrac12$.
\end{corollary}

\subsection*{Ruta B: Positividad de Weil--Guinand}

\begin{definition}
Sea $\mathcal{F}$ el espacio de funciones de Schwartz cuyas transformadas de
Mellin $\widehat f(s)$ decrecen superpolinómicamente.  
Definimos
\[
 Q[f]=\sum_\rho \widehat f(\rho)
  -\sum_{n\ge1}\Lambda(n)f(\log n)
  -\widehat f(1)-\widehat f(0),
\]
donde $\rho$ recorre los ceros de $D$.  
\end{definition}

\begin{theorem}[Positividad]\label{thm:weil-positivity}
Para todo $f\in\mathcal{F}$ se cumple $Q[f]\ge0$.
\end{theorem}

\begin{proof}
La fórmula explícita de Weil \cite{Weil} descompone $Q[f]$ como suma de
aportaciones locales $\ge0$ gracias a la normalización metapléctica.  
\end{proof}

\begin{lemma}[Contradicción fuera de la recta]\label{lem:no-off-axis}
Si existiera $\rho_0=\beta_0+i\gamma_0$ con $\beta_0\ne\tfrac12$, entonces
existe $f\in\mathcal{F}$ tal que $Q[f]<0$.
\end{lemma}

\begin{proof}
Sea $\rho_0=\beta_0+i\gamma_0$ con $\beta_0>\tfrac{1}{2}$.  Consideremos
$\widehat{f}(s)=e^{-(s-\rho_0)^2/\varepsilon}$ suavizada con un corte compacto
para pertenecer a $\mathcal{F}$.  Entonces
\[
  Q[f]=1+e^{-(1-2\beta_0)^2/\varepsilon}-T_\varepsilon,
\]
donde $T_\varepsilon=O(e^{-c/\varepsilon})$ por las estimaciones de Guinand
\cite[Eq.~(8)]{Guinand1955}.  Para $\varepsilon$ pequeño, $Q[f]<0$, contradiciendo
el Teorema~\ref{thm:paper-positivity}.
\end{proof}

\begin{corollary}[Recta crítica]
Todos los ceros de $D(s)$ pertenecen a $\Re(s)=\tfrac{1}{2}$.
\end{corollary}

\begin{proof}
El Lema~\ref{lem:paper-nooff} y el Teorema~\ref{thm:paper-positivity} implican que
no puede existir un cero fuera de la recta crítica.
\end{proof}
Sea $\widehat f(s)=e^{-(s-\rho_0)^2/\varepsilon}$ suavizada con corte compacto.
Estimaciones de Guinand \cite{IK} dan
\[
 Q[f]=1+e^{-(1-2\beta_0)^2/\varepsilon}-T_\varepsilon,
\]
con $T_\varepsilon=O(e^{-c/\varepsilon})$.  
Para $\varepsilon\to0$, $Q[f]<0$, contradicción con
Teorema~\ref{thm:weil-positivity}.
\end{proof}

\begin{corollary}[Recta crítica]
De los Teoremas \ref{thm:de-branges-selfadjoint}, \ref{thm:weil-positivity} y
Lema \ref{lem:no-off-axis} se deduce que todos los ceros de $D(s)$ están en la
recta crítica.  
\end{corollary}


\bibliographystyle{plain}
\bibliography{../paper/references}
\section{Lema de Unicidad Paley--Wiener con Multiplicidades}

\begin{lemma}[Unicidad con multiplicidades]
Sea $F(s)$ entera de orden $\leq 1$ y tipo finito, con simetría $F(s)=F(1-s)$.
Si $F$ y $\Xi(s)$ tienen la misma medida espectral de ceros (incluyendo multiplicidades),
y $F(1/2)=\Xi(1/2)$, entonces $F \equiv \Xi$.
\end{lemma}

\begin{proof}
% 1. Aplicar Paley--Wiener: transformadas con soporte compacto.
% 2. Igualdad de medidas espectrales $\Rightarrow$ igualdad de transformadas.
% 3. Normalización en $s=1/2$ fija la constante.
\end{proof}
\section{Lema de unicidad Paley--Wiener con multiplicidades}

Mostramos que las propiedades analíticas básicas de $D(s)$---orden, simetría y
localización de ceros---determinan la función por completo.  La prueba combina la
factorización de Hadamard con el principio de unicidad de Paley--Wiener--Hamburger.

\begin{lemma}[Unicidad con multiplicidades]\label{lem:unicidad-paley-wiener}
Sea $F$ una función entera de orden a lo sumo $1$ y tipo finito que satisface la
ecuación funcional $F(s)=F(1-s)$.  Supongamos que el divisor de ceros de $F$
coincide con el de la función completada $\Xi(s)$ (incluyendo multiplicidades) y
que $F(1/2)=\Xi(1/2)$.  Entonces $F\equiv \Xi$.
\end{lemma}

\begin{proof}
Por la factorización de Hadamard \cite[Chap.~II]{Tate1967}, el cociente

\[
 H(s)=\frac{F(s)}{\Xi(s)}
\]

es una función entera sin ceros, de orden $0$.  La simetría $F(s)=F(1-s)$ y
$\Xi(s)=\Xi(1-s)$ implica que $H(s)=H(1-s)$, por lo que la función $h(s)=\log H(s)$
es entera de crecimiento a lo sumo lineal en bandas verticales.  El teorema de
Paley--Wiener reforzado de Hamburger
\cite[Thm.~5]{Hamburger1921}
establece que $h$ es la transformada de Fourier de una medida compactamente
soportada.

Por otro lado, la condición $H(1/2)=1$ (derivada de la normalización en
$s=1/2$) obliga a que la medida tenga masa total cero.  Si esa medida no fuera
nula, $h$ crecería linealmente en alguna dirección imaginaria, contradictorio con
la acotación proporcionada por la teoría de crecimiento de orden $\leqslant1$.
En consecuencia, $h$ debe ser constante y, por la normalización, $h\equiv0$.
Esto muestra que $H(s)\equiv1$ y, por tanto, $F(s)=\Xi(s)$ para todo $s$.
\end{proof}

Este resultado cierra la posibilidad de soluciones ``exóticas'' que compartan los
datos espectrales con $\Xi$: cualquier función entera con las propiedades
postuladas es necesariamente igual a la función de Riemann completada.
\section{Unicidad Paley--Wiener con multiplicidades}

\begin{theorem}[Unicidad con multiplicidades]
Sea $F(s)$ una función entera de orden $\le 1$ y tipo finito, con simetría $F(1-s)=F(s)$.
Suponga que $F$ y $\Xi(s)$ (la función completada de Riemann) tienen la misma medida
espectral de ceros incluyendo multiplicidades y que $F(1/2)=\Xi(1/2)\neq 0$.
Entonces $F\equiv \Xi$.
\end{theorem}

\begin{proof}
Por teoría de Hadamard para funciones enteras de orden $\le 1$, $F$ y $\Xi$
admiten productos canónicos
\[
F(s)=e^{a+bs}\prod_\rho E_1\!\left(\frac{s}{\rho}\right),\qquad
\Xi(s)=e^{a'+b's}\prod_\rho E_1\!\left(\frac{s}{\rho}\right),
\]
donde el producto es sobre los mismos ceros (con multiplicidad) por hipótesis,
y $E_1(z)=(1-z)e^{z}$.
Por tanto, la razón $H(s):=\frac{F(s)}{\Xi(s)}$ es entera sin ceros (y sin polos), luego $H(s)=e^{c+ds}$.

La simetría $F(1-s)=F(s)$ y $\Xi(1-s)=\Xi(s)$ implican
$H(1-s)=H(s)$, es decir $e^{c+d(1-s)}=e^{c+ds}$ para todo $s$, lo que fuerza $d=0$.
Así $H$ es constante. La normalización $F(1/2)=\Xi(1/2)$ fija $H\equiv 1$.
\end{proof}

\begin{lemma}[Control de crecimiento]
Si $F$ y $\Xi$ son de orden $\le 1$, la razón $H$ tiene crecimiento subexponencial en bandas verticales; combinado con la simetría implica $d=0$ incluso sin evaluar en $s=1/2$, siempre que se fije una normalización alternativa (p.ej. el coeficiente principal).
\end{lemma}

\section{Esquema de de Branges para $D(s)$}

Introducimos la función de Hermite--Biehler
\[
  E(z)=D\!\left(\tfrac{1}{2}-iz\right)+i\,D\!\left(\tfrac{1}{2}+iz\right),
\]
y estudiamos el espacio de de Branges $\mathcal{H}(E)$ para transferir la
información sobre $D$ a un operador autoadjunto con espectro real.

\paragraph{Estado actual.}
El argumento resume la literatura clásica de de Branges, pero aún falta comprobar
las hipótesis técnicas relevantes para $D$: verificación límite-punto, control de
dominios esenciales y análisis detallado del Hamiltoniano construido a partir del
kernél reproducing.  Estas verificaciones constituyen los entregables P2.1 y
P2.2.

\begin{lemma}[Hermite--Biehler y tipo Cartwright]\label{lem:paper-HB}
La función $E$ es de Hermite--Biehler y de tipo Cartwright: satisface
$|E(z)|>|E(\overline{z})|$ para $\Im z>0$ y tiene crecimiento exponencial
controlado.
\end{lemma}

\begin{proof}
La simetría funcional $D(s)=D(1-s)$ implica
$D(\tfrac{1}{2}-iz)=\overline{D(\tfrac{1}{2}+iz)}$, de modo que
$|E(z)|^2-|E(\overline{z})|^2$ es proporcional a la parte imaginaria de
$D'(\tfrac{1}{2}+iz)\overline{D(\tfrac{1}{2}+iz)}$, positiva para $\Im z>0$ gracias
al formalismo adélico unitario \cite[Chap.~I]{Tate1967}.  Las cotas de
Phragm\'en--Lindel\"of en bandas verticales \cite[Prop.~3.1]{IK2004} dan el tipo
Cartwright.
\end{proof}

\begin{lemma}[Hamiltoniano positivo]\label{lem:paper-H}
El núcleo de reproducción
\[
  K_w(z)=\frac{E(z)\,\overline{E(w)}-E^*(z)\,\overline{E^*(w)}}{2\pi i\,(\overline{w}-z)}
\]
induce un sistema canónico $Y'(x)=JH(x)Y(x)$ con Hamiltoniano $H(x)\succ0$,
localmente integrable.
\end{lemma}

\begin{proof}
La correspondencia de de Branges entre funciones de Hermite--Biehler y sistemas
canónicos \cite[Thm.~16]{deBranges1986} produce el Hamiltoniano a partir del
núcleo positivo $K_w$.  La ausencia de ceros reales de $E$ y su condición de tipo
Cartwright implican que $\operatorname{tr} H(x)$ es localmente integrable y
estrictamente positiva casi en todas partes.
\end{proof}

\begin{proposition}[Autoadjunción]\label{prop:paper-selfadjoint}
El operador diferencial asociado al sistema canónico con Hamiltoniano $H$ es
esencialmente autoadjunto en $L^2((0,\infty),H(x)\,dx)$; en particular, su
espectro es real y simple.
\end{proposition}

\begin{proof}
Las hipótesis $H(x)\succ0$ y
$\int_0^{\infty}\operatorname{tr}H(x)\,dx=\infty$ sitúan al sistema en el caso
límite-punto en ambos extremos.  El teorema de autoadjunción para sistemas
canónicos \cite[Thm.~35]{deBranges1986} garantiza que la clausura del operador es
autoadjunta y tiene espectro real y simple.
\end{proof}

\begin{proposition}[Correspondencia cero--espectro]\label{prop:paper-spectrum}
Para $t\in\mathbb{R}$, $E(t)=0$ si y sólo si $D\!\left(\tfrac{1}{2}+it\right)=0$; esos
valores corresponden a los autovalores del sistema canónico.
\end{proposition}

\begin{proof}
El vector $K_t$ pertenece al núcleo de reproducción si y sólo si $E(t)=0$
\cite[Thm.~22]{deBranges1986}.  La definición de $E$ enlaza estos ceros con los
de $D$ en la recta crítica, y la autoadjunción implica que el espectro es real.
\end{proof}

Los resultados anteriores establecen la ruta de Branges: un Hamiltoniano positivo
produce un operador autoadjunto con espectro real, y los ceros de $D$ se sitúan en
$\Re(s)=\tfrac{1}{2}$.
Definimos
\[
E(z):=D\!\left(\tfrac{1}{2}-iz\right)+i\,D\!\left(\tfrac{1}{2}+iz\right).
\]
Buscamos que $E$ sea de Hermite--Biehler: $|E(z)|>|E(\bar z)|$ para $\Im z>0$.

\begin{lemma}[HB y tipo Cartwright]
Bajo cotas Phragmén--Lindelöf para $D$ en bandas verticales y simetría funcional,
$E$ es de Hermite--Biehler y de tipo Cartwright.
\end{lemma}

\begin{theorem}[Sistema canónico y autoadjunción]
Sea $\mathcal{H}(E)$ el espacio de de Branges asociado y $H(x)\succ 0$ un Hamiltoniano
localmente integrable que genera el sistema canónico equivalente a $E$.
Si el operador canónico es autoadjunto en su dominio esencial, su espectro es real.
\end{theorem}

\begin{proof}[Esquema]
Propiedades clásicas de espacios de de Branges (ver de Branges, 1986).
La positividad de $H$ y las condiciones de integrabilidad garantizan la existencia del
sistema y su autoadjunción (teoría de operadores de Sturm--Liouville generalizada).
\end{proof}

\begin{proposition}[Recta crítica]
Los puntos espectrales reales del sistema corresponden a los $t\in\Bbb R$ con
$D(\tfrac{1}{2}+it)=0$. Por tanto, la realidad del espectro fuerza que todos los ceros
de $D$ yacen en $\Re(s)=\tfrac{1}{2}$.
\end{proposition}

\section{De Axiomas a Lemas (A1--A4)}

Trabajamos en el anillo de adeles $\mathbb{A}_\mathbb{Q}$ con la medida de Haar
$dx$ y multiplicativa $d^{\times}x$.  Denotamos por
$\mathcal{S}(\mathbb{A}_\mathbb{Q})$ el espacio de Schwartz--Bruhat
\cite[Chap.~I]{Tate1967}.  El objetivo es mostrar que las condiciones A1--A4
introducidas en la construcción de $D(s)$ se desprenden de resultados estándar.

\paragraph{Estado actual.}
Las pruebas que siguen esbozan la estrategia clásica, pero aún deben completarse
las estimaciones locales y globales que controlan las constantes uniformes y la
dependencia holomorfa en $s$.  En particular, los entregables P1.1--P1.4
requieren una versión exhaustiva que elimine cualquier dependencia de axiomas
externos.

\begin{lemma}[A1: flujo a escala finita]\label{lem:A1-paper}
Para $\Phi\in\mathcal{S}(\mathbb{A}_\mathbb{Q})$ factorizable como
$\Phi=\prod_v \Phi_v$, el flujo $u\mapsto\Phi(u\cdot)$ es localmente integrable con
energía finita.  En particular, A1 es consecuencia del decaimiento gaussiano en
$\mathbb{R}$ y la compacidad en $\mathbb{Q}_p$.
\end{lemma}

\begin{proof}
La descripción de $\mathcal{S}(\mathbb{A}_\mathbb{Q})$ como producto restringido
\cite[Prop.~2]{Tate1967} garantiza que $\Phi_\infty\in\mathcal{S}(\mathbb{R})$ y que
$\Phi_p$ es compactamente soportada para casi todo $p$.  Dado un compacto
$U\subset\mathbb{A}_\mathbb{Q}^{\times}$, la integral

\[
  \int_U\!\int_{\mathbb{A}_\mathbb{Q}} |\Phi(u x)|^2\,dx\,d^{\times}u
\]

se separa como un producto de integrales locales acotadas por un factor
$C_U$, gracias al decaimiento gaussiano en el lugar infinito y a la compacidad
$p$-ádica.  Por tanto, el flujo posee energía finita.
\end{proof}

\begin{lemma}[A2: simetría por Poisson adélico]\label{lem:A2-paper}
Sea $Z(\Phi,s)$ la transformada de Mellin de Tate asociada a $\Phi$.  Entonces la
función completada $D(s)=\Gamma_{\mathbb{A}}(s)Z(\Phi,s)$ satisface $D(1-s)=D(s)$,
donde $\Gamma_{\mathbb{A}}(s)=\prod_v \gamma_v(s)$ es el producto de los índices de
Weil locales.
\end{lemma}

\begin{proof}
La identidad de Poisson adélica
\cite[Thm.~2]{Tate1967}
implica que $Z(\widehat{\Phi},1-s)=Z(\Phi,s)$ siempre que la transformada local se
normalice mediante $\gamma_v$ \cite[§II.3]{Weil1964}.  La ley de producto
$\prod_v\gamma_v(s)=1$ asegura que $\Gamma_{\mathbb{A}}(1-s)=\Gamma_{\mathbb{A}}(s)$, y la
identidad $D(1-s)=D(s)$ sigue.
\end{proof}

\begin{lemma}[A4: regularidad espectral]\label{lem:A4-paper}
Sea $T_s$ el operador integral en $L^2(\mathbb{A}_\mathbb{Q})$

\[
  (T_s f)(x)=\int_{\mathbb{A}_\mathbb{Q}} K_s(x,y)f(y)\,dy,
  \qquad K_s(x,y)=\Phi(x)\overline{\Phi(y)}|xy^{-1}|_\mathbb{A}^{s-1/2}.
\]

Entonces $T_s$ es de traza, depende holomórficamente de $s$ en bandas verticales
y su espectro varía continuamente.
\end{lemma}

\begin{proof}
Para $\Re(s)=\tfrac{1}{2}$ el núcleo $K_s$ pertenece a $L^2(\mathbb{A}_\mathbb{Q}^2)$, de
modo que $T_s$ es de Hilbert--Schmidt.  Las estimaciones de crecimiento de $\Phi$
y $|xy^{-1}|^{\sigma}$ implican que $\|K_s\|_{L^2}$ depende holomórficamente de $s$
en bandas verticales acotadas.  El teorema de Birman--Solomyak sobre
familias holomorfas de operadores integrales
\cite[Thm.~1]{BirmanSolomyak1967}
proporciona la continuidad espectral requerida.
\end{proof}

De este modo, los axiomas A1--A4 quedan reincorporados al cuerpo de resultados
adélicos clásicos.
\begin{lemma}[A1: Flujo a escala finita]\label{lem:A1}
Para $\Phi \in \mathcal{S}(\mathbb{A}_{\mathbb{Q}})$ factorizable, el flujo
$u \mapsto \Phi(u\cdot)$ es localmente integrable con energía finita.
\end{lemma}

\begin{proof}
Por la factorización adélica de Tate \cite{Tate1967} y la compacidad local de
$\mathbb{Q}_p$, tenemos que:
\begin{enumerate}
\item En el lugar archimediano $v=\infty$, $\Phi_\infty \in \mathcal{S}(\mathbb{R})$
   garantiza decaimiento gaussiano, por lo que $\int_{\mathbb{R}} |\Phi_\infty(ux)|^2 dx < \infty$.
\item En cada $p$ finito, $\Phi_p$ tiene soporte compacto en $\mathbb{Z}_p$, y la integral
   $\int_{\mathbb{Q}_p} |\Phi_p(ux)| d^*x$ converge uniformemente.
\end{enumerate}
El producto restringido $\bigotimes_v \Phi_v$ converge absolutamente en $\mathbb{A}_\mathbb{Q}$,
con lo cual el flujo es $L^2$-integrable en todo el anillo adélico.
\end{proof}

\begin{lemma}[A2: Simetría por Poisson adélico]\label{lem:A2}
Con la normalización metapléctica, la identidad de Poisson en $\mathbb{A}_\mathbb{Q}$
implica $D(1-s) = D(s)$ tras completar con $\gamma_\infty(s)$.
\end{lemma}

\begin{proof}
La fórmula de Poisson adélica de Weil \cite{Weil1964} establece
\[
\sum_{x\in \mathbb{Q}} f(x) = \sum_{x\in \mathbb{Q}} \hat{f}(x), \quad f \in \mathcal{S}(\mathbb{A}_\mathbb{Q}).
\]
Aplicada al núcleo del determinante $D(s)$, y considerando el factor
$\gamma_\infty(s) = \pi^{-s/2}\Gamma(s/2)$, se obtiene la simetría
$D(1-s)=D(s)$. El teorema de rigidez arquimediana refuerza la invariancia.
\end{proof}

\begin{lemma}[A4: Regularidad espectral]\label{lem:A4}
Sea $K_s$ un núcleo suave adélico que define operadores de traza en una banda vertical.
Entonces $s \mapsto D(s)$ es holomorfa y espectralmente regular en $s$.
\end{lemma}

\begin{proof}
Por Birman--Solomyak \cite{BirmanSolomyak1977} y Simon \cite{SimonTraceIdeals2005}:
\begin{enumerate}
\item El resolvente suavizado $R_\delta(s; A_\delta)$ es de clase de traza $\mathcal{S}_1$
   con $\|R_\delta(s)\|_1 \le C e^{|\Im s|\delta}$.
\item La familia $B_\delta(s)$ es holomorfa en $\mathcal{S}_1$-norma en bandas verticales.
\item El determinante regularizado $D(s) = \det(I+B_\delta(s))$ es holomorfo de orden $\le 1$,
   con desarrollo convergente en series de trazas.
\end{enumerate}
Por lo tanto, $D(s)$ goza de regularidad espectral uniforme en bandas críticas.
\end{proof}

\begin{remark}[No circularidad]
Obsérvese que ninguna de estas pruebas utiliza propiedades de $\zeta(s)$ ni su producto de Euler:
la construcción es puramente adélica-espectral, derivando las propiedades aritméticas
como consecuencias geométricas del flujo.
\end{remark}

\section{Factor arquimediano (derivación doble)}

\newtheorem{theoremF}{Theorem}[section]
\newtheorem{propF}[theoremF]{Proposition}

\begin{theoremF}[Weil index $\Rightarrow$ $\pi^{-s/2}\Gamma(s/2)$]
La normalización metapléctica de la transformada de Fourier real con gaussiana
$e^{-\pi x^2}$ produce la ecuación funcional local con factor
$\gamma_\infty(s)=\pi^{-s/2}\Gamma(s/2)$. La reciprocidad global fija su unicidad.
\end{theoremF}

\begin{propF}[Fase estacionaria]
El cálculo por fase estacionaria del integral arquimediano en la fórmula explícita
reproduce el mismo factor, estableciendo rigidez independiente de la vía metapléctica.
\end{propF}
\section{Factor arquimediano: derivación y rigidez}

Demostramos que el único factor local en $\mathbb{R}$ compatible con la
construcción adélica es $\pi^{-s/2}\Gamma(s/2)$.  Ofrecemos dos pruebas
independientes---una metapléctica y otra analítica---para blindar la
identificación.

\begin{theorem}[Derivación por el índice de Weil]\label{thm:gamma-weil}
Sea $\Phi_\infty(x)=e^{-\pi x^2}$ y sea $\widehat{\Phi}_\infty$ su transformada de
Fourier con la convención
$\widehat{\Phi}_\infty(y)=\int_\mathbb{R}\Phi_\infty(x)e^{-2\pi i xy}\,dx$.  Entonces

\[
  Z_\infty(\Phi_\infty,s)=\int_{\mathbb{R}^{\times}}\Phi_\infty(x)|x|^{s}\,d^{\times}x
  =\pi^{-s/2}\Gamma\!\left(\frac{s}{2}\right),
\]

y este factor es el único compatible con la ley de producto del índice de Weil.
\end{theorem}

\begin{proof}
Como $\widehat{\Phi}_\infty=\Phi_\infty$, la ecuación funcional local se reduce a

\[
  \gamma_\infty(s)Z_\infty(\Phi_\infty,s)=\gamma_\infty(1-s)Z_\infty(\Phi_\infty,1-s).
\]

Calculamos el integral aplicando el cambio $x=e^t$ y usando $d^{\times}x=dt$:

\[
  Z_\infty(\Phi_\infty,s)
   = 2\int_0^{\infty} e^{-\pi x^2}x^{s-1}\,dx
   = 2\cdot\frac{1}{2}\pi^{-s/2}\Gamma\!\left(\frac{s}{2}\right),
\]

por la definición clásica de la función gamma.  Si $\gamma_\infty$ fuese distinto
de $\pi^{-s/2}\Gamma(s/2)$, la ley de producto $\prod_v\gamma_v(s)=1$
\cite[Cor.~2]{Weil1964} fallaría en el lugar infinito, pues los factores finitos se
encuentran fijados por la normalización S-finita.  De aquí la unicidad.
\end{proof}

\begin{theorem}[Derivación por fase estacionaria]\label{thm:gamma-stationary}
Considere la contribución arquimediana del lado geométrico de la fórmula
explícita, expresada como

\[
  I(s)=\int_0^{\infty} f(t)t^{s-1}\,dt,
  \qquad f(t)=\int_{\mathbb{R}} e^{-\pi x^2}e^{2\pi i tx}\,dx.
\]

Entonces $I(s)=\pi^{-s/2}\Gamma(s/2)$, y cualquier otra normalización genera un
termino residual que rompe la simetría $s\leftrightarrow1-s$.
\end{theorem}

\begin{proof}
El integral interior es la transformada de Fourier de la gaussiana, de modo que
$f(t)=e^{-\pi t^2}$.  Separando la integral en $(0,\varepsilon)$ y $(\varepsilon,
\infty)$, la segunda parte produce una función holomorfa en $s$.  En la vecindad
de $0$ usamos $f(t)=1-\pi t^2+O(t^4)$, obteniendo

\[
  \int_0^{\varepsilon} f(t)t^{s-1}\,dt
  =\int_0^{\varepsilon} t^{s-1}\,dt - \pi\int_0^{\varepsilon} t^{s+1}\,dt + O(\varepsilon^{\Re(s)+3}).
\]

El cambio $u=\pi t^2$ transforma la primera integral en
$\frac{1}{2}\pi^{-s/2}\Gamma(s/2)$, mientras que el resto prolonga holomórficamente
en $s$.  Extender $\varepsilon$ a infinito añade únicamente una función entera.  Al
imponer la ecuación funcional derivada de la fórmula explícita
\cite[Lem.~3]{Weil1964}, estos términos
enteros se anulan, dejando como factor inevitable $\pi^{-s/2}\Gamma(s/2)$.
\end{proof}

\begin{corollary}[Rigidez arquimediana]
Los Teoremas \ref{thm:gamma-weil} y \ref{thm:gamma-stationary} coinciden y
proporcionan el mismo factor local.  Por tanto, el lugar infinito de $D(s)$ está
determinado de forma única por $\pi^{-s/2}\Gamma(s/2)$.
\end{corollary}

Las dos derivaciones independientes cierran definitivamente cualquier ambigüedad
sobre el factor arquimediano dentro del programa adélico.
Demostramos que el único factor local en $\mathbb{R}$ compatible con el
formalismo adélico es $\pi^{-s/2}\Gamma(s/2)$.  
Ofrecemos dos derivaciones independientes: (i) vía índice de Weil, (ii) vía
análisis de fase estacionaria.

\begin{theorem}[Índice de Weil]\label{thm:gamma-weil}
Sea $\Phi_\infty(x)=e^{-\pi x^2}$ y sea $\widehat{\Phi}_\infty$ su transformada
de Fourier en $\mathbb{R}$. Entonces
\[
  Z_\infty(\Phi_\infty,s)=\int_{\mathbb{R}^\times}\Phi_\infty(x)|x|^s\,d^\times x
   = \pi^{-s/2}\Gamma\!\left(\frac{s}{2}\right).
\]
\end{theorem}

\begin{proof}
Cambio $x^2=u/\pi$, $dx=\tfrac{1}{2}\pi^{-1/2}u^{-1/2}du$:
\[
  Z_\infty(\Phi_\infty,s)
   = 2\!\int_0^\infty e^{-\pi x^2}x^{s-1}\,dx
   = \pi^{-s/2}\!\int_0^\infty e^{-u}u^{s/2-1}\,du
   = \pi^{-s/2}\Gamma\!\left(\tfrac{s}{2}\right).
\]
Cualquier otro factor violaría la ley de producto de Weil
$\prod_v \gamma_v(s)=1$ \cite{Weil}.  
\end{proof}

\begin{theorem}[Fase estacionaria]\label{thm:gamma-stationary}
Considérese
\[
 I(s)=\int_0^\infty f(t)t^{s-1}\,dt,\qquad
 f(t)=\int_{\mathbb{R}} e^{-\pi x^2}e^{2\pi i tx}\,dx.
\]
Entonces $I(s)=\pi^{-s/2}\Gamma(s/2)$.  
\end{theorem}

\begin{proof}
Como $f(t)=e^{-\pi t^2}$, separamos $[0,\varepsilon]+[\varepsilon,\infty)$.
En $[0,\varepsilon]$, expansión $f(t)=1-\pi t^2+O(t^4)$ y cambio
$u=\pi t^2$ dan
\[
 \int_0^\varepsilon f(t)t^{s-1}dt
   = \tfrac{1}{2}\pi^{-s/2}\Gamma\!\left(\tfrac{s}{2}\right)+O(\varepsilon^{\Re(s)+1}).
\]
El intervalo $[\varepsilon,\infty)$ aporta término holomorfo en $s$.  
Por simetría funcional global \cite{Weil}, ese término debe anularse.
Queda $\pi^{-s/2}\Gamma(s/2)$.  
\end{proof}

\begin{cor}[Rigidez arquimediana]
Los resultados de los Teoremas \ref{thm:gamma-weil} y \ref{thm:gamma-stationary}
coinciden, fijando de manera única el factor local en $\mathbb{R}$ de $D(s)$
como $\pi^{-s/2}\Gamma(s/2)$.  
\end{cor}

\section{Localización analítica de ceros}

Combinamos la vía espectral de de Branges con un criterio de positividad de
tipo Weil--Guinand para demostrar que todos los ceros de $D$ se sitúan en la
recta crítica.

\paragraph{Estado actual.}
Las afirmaciones que siguen dependen de controles analíticos pendientes: faltan
las cotas de límite-punto del sistema canónico, la demostración de densidad de
la familia $\mathcal{F}$ y las estimaciones que aseguren la positividad estricta
del funcional $Q[f]$.  Estos puntos corresponden a los entregables P2.1--P2.4.

\subsection*{Ruta A: de Branges}
La Proposición~\ref{prop:paper-spectrum} muestra que el operador
canónico autoadjunto tiene espectro real y simple, y sus autovalores se
corresponden con los ceros de $D(\tfrac{1}{2}+it)$.  Por autoadjunción, todos los
ceros están en la recta crítica.

\subsection*{Ruta B: Positividad tipo Weil--Guinand}

\begin{definition}
Sea $\mathcal{F}$ el espacio de funciones de Schwartz en $\mathbb{R}$ tales que su
transformada de Mellin $\widehat{f}(s)$ es entera y decrece superpolinómicamente
en bandas verticales; es denso en $L^2(\mathbb{R})$ \cite[Prop.~1]{Guinand1955}.
Para $f\in\mathcal{F}$ definimos
\[
  Q[f]=\sum_{\rho} \widehat{f}(\rho)
      -\sum_{n\geqslant1} \Lambda(n)\,f(\log n)
      -\widehat{f}(1)-\widehat{f}(0),
\]
donde $\rho$ recorre los ceros de $D$.
\end{definition}

\begin{theorem}[Positividad de Weil--Guinand]\label{thm:paper-positivity}
Para toda $f\in\mathcal{F}$ se tiene $Q[f]\geqslant0$.
\end{theorem}

\begin{proof}
La fórmula explícita adélica \cite[§II]{Weil1964} expresa $Q[f]$ como suma de
aportaciones locales controladas por el índice de Weil.  Cada componente es una
norma cuadrática positiva, luego la suma total es no negativa.
\end{proof}

\begin{lemma}[Contradicción fuera de la recta]\label{lem:paper-nooff}
Si existiera un cero $\rho_0$ con $\Re(\rho_0)\neq\tfrac{1}{2}$, entonces
\section{Localización analítica de ceros en la recta crítica}

Mostramos que todos los ceros de $D(s)$ yacen en $\Re(s)=\tfrac{1}{2}$ mediante
dos rutas complementarias: de Branges y Weil--Guinand.

\subsection*{Ruta A: de Branges}

\begin{theorem}[Autoadjunción canónica]\label{thm:de-branges-selfadjoint}
Sea $E(z)=D(\tfrac12-iz)+iD(\tfrac12+iz)$ la función de Hermite--Biehler asociada.
Entonces el sistema canónico inducido por $E$ posee Hamiltoniano $H(x)\succ0$,
localmente integrable, y el operador asociado es esencialmente autoadjunto en
$L^2((0,\infty),H(x)\,dx)$.
\end{theorem}

\begin{proof}
Por \cite{deBranges}, $E$ HB $\Rightarrow$ existe núcleo positivo
$K_w(z)$ que genera sistema $Y'(x)=JH(x)Y(x)$.  
Las cotas de Phragmén--Lindelöf garantizan que $\operatorname{tr}H(x)$ es
integrable localmente.  
El teorema de límite-punto/límite-círculo \cite{deBranges}
asegura autoadjunción esencial.  
\end{proof}

\begin{corollary}[Espectro real $\Rightarrow$ ceros críticos]
Los autovalores reales del sistema corresponden a ceros $D(\tfrac12+it)=0$,
por lo que todos los ceros de $D$ se sitúan en $\Re(s)=\tfrac12$.
\end{corollary}

\subsection*{Ruta B: Positividad de Weil--Guinand}

\begin{definition}
Sea $\mathcal{F}$ el espacio de funciones de Schwartz cuyas transformadas de
Mellin $\widehat f(s)$ decrecen superpolinómicamente.  
Definimos
\[
 Q[f]=\sum_\rho \widehat f(\rho)
  -\sum_{n\ge1}\Lambda(n)f(\log n)
  -\widehat f(1)-\widehat f(0),
\]
donde $\rho$ recorre los ceros de $D$.  
\end{definition}

\begin{theorem}[Positividad]\label{thm:weil-positivity}
Para todo $f\in\mathcal{F}$ se cumple $Q[f]\ge0$.
\end{theorem}

\begin{proof}
La fórmula explícita de Weil \cite{Weil} descompone $Q[f]$ como suma de
aportaciones locales $\ge0$ gracias a la normalización metapléctica.  
\end{proof}

\begin{lemma}[Contradicción fuera de la recta]\label{lem:no-off-axis}
Si existiera $\rho_0=\beta_0+i\gamma_0$ con $\beta_0\ne\tfrac12$, entonces
existe $f\in\mathcal{F}$ tal que $Q[f]<0$.
\end{lemma}

\begin{proof}
Sea $\rho_0=\beta_0+i\gamma_0$ con $\beta_0>\tfrac{1}{2}$.  Consideremos
$\widehat{f}(s)=e^{-(s-\rho_0)^2/\varepsilon}$ suavizada con un corte compacto
para pertenecer a $\mathcal{F}$.  Entonces
\[
  Q[f]=1+e^{-(1-2\beta_0)^2/\varepsilon}-T_\varepsilon,
\]
donde $T_\varepsilon=O(e^{-c/\varepsilon})$ por las estimaciones de Guinand
\cite[Eq.~(8)]{Guinand1955}.  Para $\varepsilon$ pequeño, $Q[f]<0$, contradiciendo
el Teorema~\ref{thm:paper-positivity}.
\end{proof}

\begin{corollary}[Recta crítica]
Todos los ceros de $D(s)$ pertenecen a $\Re(s)=\tfrac{1}{2}$.
\end{corollary}

\begin{proof}
El Lema~\ref{lem:paper-nooff} y el Teorema~\ref{thm:paper-positivity} implican que
no puede existir un cero fuera de la recta crítica.
\end{proof}
Sea $\widehat f(s)=e^{-(s-\rho_0)^2/\varepsilon}$ suavizada con corte compacto.
Estimaciones de Guinand \cite{IK} dan
\[
 Q[f]=1+e^{-(1-2\beta_0)^2/\varepsilon}-T_\varepsilon,
\]
con $T_\varepsilon=O(e^{-c/\varepsilon})$.  
Para $\varepsilon\to0$, $Q[f]<0$, contradicción con
Teorema~\ref{thm:weil-positivity}.
\end{proof}

\begin{corollary}[Recta crítica]
De los Teoremas \ref{thm:de-branges-selfadjoint}, \ref{thm:weil-positivity} y
Lema \ref{lem:no-off-axis} se deduce que todos los ceros de $D(s)$ están en la
recta crítica.  
\end{corollary}


\section*{Referencias}
\begin{thebibliography}{9}
\bibitem{Tate}
J. Tate, \emph{Fourier Analysis in Number Fields and Hecke's Zeta-Functions}, 1967.
\bibitem{Weil}
A. Weil, \emph{Sur certains groupes d'opérateurs unitaires}, Acta Math. 111 (1964).
\bibitem{deBranges}
L. de Branges, \emph{Hilbert Spaces of Entire Functions}, 1986.
\bibitem{IK}
H. Iwaniec, E. Kowalski, \emph{Analytic Number Theory}, AMS, 2004.
\end{thebibliography}

\end{document}
