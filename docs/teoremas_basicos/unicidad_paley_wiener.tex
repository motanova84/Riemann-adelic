\section{Lema de Unicidad Paley--Wiener con Multiplicidades}

\begin{lemma}[Unicidad con multiplicidades]
Sea $F(s)$ entera de orden $\leq 1$ y tipo finito, con simetría $F(s)=F(1-s)$.
Si $F$ y $\Xi(s)$ tienen la misma medida espectral de ceros (incluyendo multiplicidades),
y $F(1/2)=\Xi(1/2)$, entonces $F \equiv \Xi$.
\end{lemma}

\begin{proof}
% 1. Aplicar Paley--Wiener: transformadas con soporte compacto.
% 2. Igualdad de medidas espectrales $\Rightarrow$ igualdad de transformadas.
% 3. Normalización en $s=1/2$ fija la constante.
\end{proof}
\section{Lema de unicidad Paley--Wiener con multiplicidades}

Mostramos que las propiedades analíticas básicas de $D(s)$---orden, simetría y
localización de ceros---determinan la función por completo.  La prueba combina la
factorización de Hadamard con el principio de unicidad de Paley--Wiener--Hamburger.

\begin{lemma}[Unicidad con multiplicidades]\label{lem:unicidad-paley-wiener}
Sea $F$ una función entera de orden a lo sumo $1$ y tipo finito que satisface la
ecuación funcional $F(s)=F(1-s)$.  Supongamos que el divisor de ceros de $F$
coincide con el de la función completada $\Xi(s)$ (incluyendo multiplicidades) y
que $F(1/2)=\Xi(1/2)$.  Entonces $F\equiv \Xi$.
\end{lemma}

\begin{proof}
Por la factorización de Hadamard \cite[Chap.~II]{Tate1967}, el cociente

\[
 H(s)=\frac{F(s)}{\Xi(s)}
\]

es una función entera sin ceros, de orden $0$.  La simetría $F(s)=F(1-s)$ y
$\Xi(s)=\Xi(1-s)$ implica que $H(s)=H(1-s)$, por lo que la función $h(s)=\log H(s)$
es entera de crecimiento a lo sumo lineal en bandas verticales.  El teorema de
Paley--Wiener reforzado de Hamburger
\cite[Thm.~5]{Hamburger1921}
establece que $h$ es la transformada de Fourier de una medida compactamente
soportada.

Por otro lado, la condición $H(1/2)=1$ (derivada de la normalización en
$s=1/2$) obliga a que la medida tenga masa total cero.  Si esa medida no fuera
nula, $h$ crecería linealmente en alguna dirección imaginaria, contradictorio con
la acotación proporcionada por la teoría de crecimiento de orden $\leqslant1$.
En consecuencia, $h$ debe ser constante y, por la normalización, $h\equiv0$.
Esto muestra que $H(s)\equiv1$ y, por tanto, $F(s)=\Xi(s)$ para todo $s$.
\end{proof}

Este resultado cierra la posibilidad de soluciones ``exóticas'' que compartan los
datos espectrales con $\Xi$: cualquier función entera con las propiedades
postuladas es necesariamente igual a la función de Riemann completada.
\section{Unicidad Paley--Wiener con multiplicidades}

\begin{theorem}[Unicidad Paley–Wiener–Hamburger con multiplicidades]
\label{thm:paley-wiener-uniqueness}
Sea $D(s)$ la función determinante construida en el marco adélico S-finito. Supongamos:
\begin{enumerate}
\item $D(s)$ es entera de orden $\leq 1$ y tipo finito;
\item $D(1-s) = D(s)$ (simetría funcional);
\item $\lim_{\Re(s)\to+\infty} \log D(s) = 0$ (normalización);
\item La medida espectral de ceros de $D(s)$ coincide con la de $\Xi(s)$, 
incluyendo multiplicidades.
\end{enumerate}
Entonces $D(s) \equiv \Xi(s)$.
\end{theorem}

\begin{proof}
\textbf{Paso 1 (Factorización de Hadamard).}  
Por Hadamard (1893), toda función entera de orden $\leq 1$ admite producto canónico:
\[
D(s) = e^{a_D + b_D s} \prod_{\rho}\!E_1\!\left(\tfrac{s}{\rho}\right), 
\qquad 
\Xi(s) = e^{a_\Xi + b_\Xi s} \prod_{\rho}\!E_1\!\left(\tfrac{s}{\rho}\right),
\]
donde $E_1(z) = (1-z)e^z$ y el producto es sobre los mismos ceros $\rho$ con multiplicidad.

\textbf{Paso 2 (Cociente auxiliar).}  
Definimos $H(s) := \tfrac{D(s)}{\Xi(s)} = e^{c + ds}$, con $c=a_D-a_\Xi$, $d=b_D-b_\Xi$.  
$H(s)$ es entera sin ceros ni polos.

\textbf{Paso 3 (Simetría funcional).}  
De $D(1-s)=D(s)$ y $\Xi(1-s)=\Xi(s)$ se obtiene
\[
H(1-s) = H(s) \implies e^{c+d(1-s)} = e^{c+ds} \implies d=0.
\]
Por tanto $H(s)=e^c$ es constante.

\textbf{Paso 4 (Normalización asintótica).}  
El límite $\lim_{\Re(s)\to+\infty} \log D(s)=0$ fuerza $b_D=0$. Como se sabe que $b_\Xi=0$, resulta $d=0$ (ya deducido). Además, la condición elimina la constante exponencial, fijando $c=0$.

\textbf{Conclusión.}  
$H(s)\equiv 1$, es decir $D(s)\equiv \Xi(s)$.
\end{proof}

\begin{lemma}[Control de crecimiento por Phragmén–Lindelöf]\label{lem:phragmen}
Para funciones enteras de orden $\leq 1$ en bandas verticales, el principio de Phragmén–Lindelöf (1908) garantiza que la condición $\lim_{\Re(s)\to+\infty}\log F(s)=0$ elimina términos lineales en la factorización de Hadamard, reforzando la unicidad.
\end{lemma}

\begin{proposition}[Clase determinante Paley–Wiener]\label{prop:pw-class}
$D(s)$ pertenece a la clase determinante de Paley–Wiener:
\begin{enumerate}
\item Tipo exponencial $\leq \sigma$ para algún $\sigma > 0$;
\item Cuadrado-integrable en líneas verticales: $\int_{-\infty}^\infty |D(\sigma+it)|^2 dt < \infty$;
\item Determinada únicamente por su medida espectral de ceros.
\end{enumerate}
\end{proposition}

\begin{proof}
El tipo exponencial está controlado por la construcción del resolvente suavizado.  
La integrabilidad cuadrática proviene de la pertenencia de $B_\delta(s)$ a la clase de traza.  
La unicidad es consecuencia directa del Teorema~\ref{thm:paley-wiener-uniqueness}.
\end{proof}

\begin{remark}[Referencias clásicas]
\begin{itemize}
\item Hadamard (1893): factorización de funciones enteras.  
\item Phragmén–Lindelöf (1908): control de crecimiento en bandas.  
\item Paley–Wiener (1934): unicidad en clases determinantes.  
\item Hamburger (1921): simetría funcional y unicidad.  
\end{itemize}
\end{remark}
