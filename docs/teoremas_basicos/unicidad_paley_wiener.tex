\section{Lema de Unicidad Paley--Wiener con Multiplicidades}

\begin{lemma}[Unicidad con multiplicidades]
Sea $F(s)$ entera de orden $\leq 1$ y tipo finito, con simetría $F(s)=F(1-s)$.
Si $F$ y $\Xi(s)$ tienen la misma medida espectral de ceros (incluyendo multiplicidades),
y $F(1/2)=\Xi(1/2)$, entonces $F \equiv \Xi$.
\end{lemma}

\begin{proof}
% 1. Aplicar Paley--Wiener: transformadas con soporte compacto.
% 2. Igualdad de medidas espectrales $\Rightarrow$ igualdad de transformadas.
% 3. Normalización en $s=1/2$ fija la constante.
\end{proof}
\section{Lema de unicidad Paley--Wiener con multiplicidades}

Mostramos que las propiedades analíticas básicas de $D(s)$---orden, simetría y
localización de ceros---determinan la función por completo.  La prueba combina la
factorización de Hadamard con el principio de unicidad de Paley--Wiener--Hamburger.

\begin{lemma}[Unicidad con multiplicidades]\label{lem:unicidad-paley-wiener}
Sea $F$ una función entera de orden a lo sumo $1$ y tipo finito que satisface la
ecuación funcional $F(s)=F(1-s)$.  Supongamos que el divisor de ceros de $F$
coincide con el de la función completada $\Xi(s)$ (incluyendo multiplicidades) y
que $F(1/2)=\Xi(1/2)$.  Entonces $F\equiv \Xi$.
\end{lemma}

\begin{proof}
Por la factorización de Hadamard \cite[Chap.~II]{Tate1967}, el cociente

\[
 H(s)=\frac{F(s)}{\Xi(s)}
\]

es una función entera sin ceros, de orden $0$.  La simetría $F(s)=F(1-s)$ y
$\Xi(s)=\Xi(1-s)$ implica que $H(s)=H(1-s)$, por lo que la función $h(s)=\log H(s)$
es entera de crecimiento a lo sumo lineal en bandas verticales.  El teorema de
Paley--Wiener reforzado de Hamburger
\cite[Thm.~5]{Hamburger1921}
establece que $h$ es la transformada de Fourier de una medida compactamente
soportada.

Por otro lado, la condición $H(1/2)=1$ (derivada de la normalización en
$s=1/2$) obliga a que la medida tenga masa total cero.  Si esa medida no fuera
nula, $h$ crecería linealmente en alguna dirección imaginaria, contradictorio con
la acotación proporcionada por la teoría de crecimiento de orden $\leqslant1$.
En consecuencia, $h$ debe ser constante y, por la normalización, $h\equiv0$.
Esto muestra que $H(s)\equiv1$ y, por tanto, $F(s)=\Xi(s)$ para todo $s$.
\end{proof}

Este resultado cierra la posibilidad de soluciones ``exóticas'' que compartan los
datos espectrales con $\Xi$: cualquier función entera con las propiedades
postuladas es necesariamente igual a la función de Riemann completada.
\section{Unicidad Paley--Wiener con multiplicidades}

\begin{theorem}[Unicidad con multiplicidades]
Sea $F(s)$ una función entera de orden $\le 1$ y tipo finito, con simetría $F(1-s)=F(s)$.
Suponga que $F$ y $\Xi(s)$ (la función completada de Riemann) tienen la misma medida
espectral de ceros incluyendo multiplicidades y que $F(1/2)=\Xi(1/2)\neq 0$.
Entonces $F\equiv \Xi$.
\end{theorem}

\begin{proof}
Por teoría de Hadamard para funciones enteras de orden $\le 1$, $F$ y $\Xi$
admiten productos canónicos
\[
F(s)=e^{a+bs}\prod_\rho E_1\!\left(\frac{s}{\rho}\right),\qquad
\Xi(s)=e^{a'+b's}\prod_\rho E_1\!\left(\frac{s}{\rho}\right),
\]
donde el producto es sobre los mismos ceros (con multiplicidad) por hipótesis,
y $E_1(z)=(1-z)e^{z}$.
Por tanto, la razón $H(s):=\frac{F(s)}{\Xi(s)}$ es entera sin ceros (y sin polos), luego $H(s)=e^{c+ds}$.

La simetría $F(1-s)=F(s)$ y $\Xi(1-s)=\Xi(s)$ implican
$H(1-s)=H(s)$, es decir $e^{c+d(1-s)}=e^{c+ds}$ para todo $s$, lo que fuerza $d=0$.
Así $H$ es constante. La normalización $F(1/2)=\Xi(1/2)$ fija $H\equiv 1$.
\end{proof}

\begin{lemma}[Control de crecimiento]
Si $F$ y $\Xi$ son de orden $\le 1$, la razón $H$ tiene crecimiento subexponencial en bandas verticales; combinado con la simetría implica $d=0$ incluso sin evaluar en $s=1/2$, siempre que se fije una normalización alternativa (p.ej. el coeficiente principal).
\end{lemma}
