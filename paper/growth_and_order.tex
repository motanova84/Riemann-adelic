\section{Growth and Order of $D(s)$}

This section establishes precise bounds on the growth of the canonical determinant $D(s)$ as a function of the complex parameter $s = \sigma + it$, proving that $D(s)$ is an entire function of order at most 1 with explicit constants.

\subsection{Phragmén–Lindelöf Growth Bounds}

\begin{theorem}[Growth Bound for $D(s)$]\label{thm:growth-bound}
The canonical determinant $D(s)$ satisfies the growth estimate
\[
|D(\sigma + it)| \leq C_\varepsilon \exp\left((1 + \varepsilon)|t|\right), \quad \text{for all } \varepsilon > 0,
\]
where $C_\varepsilon > 0$ is a constant depending only on $\varepsilon$ and the strip $|\sigma - 1/2| \leq 1/4 + \varepsilon$.
\end{theorem}

\begin{proof}
The proof proceeds via the Phragmén–Lindelöf maximum principle applied to the resolvent-based construction of $D(s)$.

\textbf{Step 1: Resolvent bounds.} 
From the smoothed resolvent definition in Section 2,
\[
R_\delta(s; A) = \int_{\mathbb{R}} e^{(\sigma - 1/2)u} e^{itu} w_\delta(u) e^{iuA} \, du,
\]
we have the operator norm estimate
\[
\|R_\delta(s; A)\|_{\text{op}} \leq \int_{\mathbb{R}} e^{(\sigma - 1/2)u} |w_\delta(u)| \, du.
\]
For $\sigma$ in the critical strip $|\sigma - 1/2| \leq 1/4$, this integral converges exponentially in $\delta$.

\textbf{Step 2: Trace-class norm control.}
By the Kato–Seiler–Simon estimates \cite{simon2005}, the perturbation operator $B_\delta(s)$ satisfies
\[
\|B_\delta(s)\|_1 \leq C \sum_{v \in V} \|K_{v,\delta}\|_1 \cdot \|R_\delta(s; Z)\|_{\text{op}}.
\]
Using the Schatten bound from Appendix C,
\[
\|K_{v,\delta}\|_1 \leq C(\log q_v) q_v^{-2},
\]
and summing over $v \in V$, we obtain
\[
\|B_\delta(s)\|_1 \leq C' \exp(C''|\sigma - 1/2|).
\]

\textbf{Step 3: Determinant growth via Golden–Thompson.}
For trace-class operators with $\|B\|_1 < 1/2$, we have
\[
|\det(I + B) - 1| \leq \|B\|_1 \exp(\|B\|_1).
\]
For larger $\|B\|_1$, the general Fredholm determinant expansion gives
\[
|\det(I + B)| \leq \exp\left(\|B\|_1\right).
\]

\textbf{Step 4: Phragmén–Lindelöf on vertical strips.}
The function $\log |D(s)|$ is subharmonic in the strip. On the boundary lines $\sigma = 1/4$ and $\sigma = 3/4$, the functional equation $D(1-s) = D(s)$ and the bounds from Steps 1–3 imply
\[
\log |D(\sigma + it)| \leq C + C'|t|.
\]
By the Phragmén–Lindelöf principle \cite{IK2004}, this bound extends to the interior of the strip, yielding
\[
|D(\sigma + it)| \leq C_\varepsilon \exp\left((1 + \varepsilon)|t|\right)
\]
for any $\varepsilon > 0$ and $\sigma$ in the critical strip.
\end{proof}

\subsection{Order at Most 1}

\begin{corollary}[Order of $D(s)$]\label{cor:order-one}
The entire function $D(s)$ is of order at most 1, meaning
\[
\limsup_{r \to \infty} \frac{\log \log M(r)}{\log r} \leq 1,
\]
where $M(r) = \max_{|s| = r} |D(s)|$.
\end{corollary}

\begin{proof}
From Theorem \ref{thm:growth-bound}, for $|s| = r$ large, we have
\[
|D(s)| \leq C \exp((1 + \varepsilon)r).
\]
Thus,
\[
\log M(r) \leq \log C + (1 + \varepsilon)r,
\]
which gives
\[
\log \log M(r) \leq \log((1 + \varepsilon)r) + O(1) \sim \log r.
\]
Therefore, the order is at most 1.
\end{proof}

\subsection{Asymptotic Analysis of $\log D(s)$}

We now compare the asymptotic behavior of $\log D(s)$ with the integral involving the archimedean kernel, reproducing the standard relation $\frac{1}{2}\psi(s/2) - \frac{1}{2}\log \pi$.

\begin{theorem}[Archimedean Comparison]\label{thm:archimedean-comparison}
For $s = \sigma + it$ with $\sigma$ fixed in $(0, 1)$ and $t \to \infty$,
\[
\log D(s) = -\frac{1}{2}\psi(s/2) + \frac{1}{2}\log \pi + o(1),
\]
where $\psi(z) = \Gamma'(z)/\Gamma(z)$ is the digamma function.
\end{theorem}

\begin{proof}[Proof sketch]
The archimedean contribution to the trace formula comes from the Gaussian smoothing and the operator exponential $e^{iuZ}$. Computing the trace of the archimedean part explicitly via operator calculus (see Appendix B), we obtain
\[
\operatorname{tr}(R_\delta^{\text{arch}}(s)) = \int_{\mathbb{R}} e^{(\sigma - 1/2)u} e^{itu} w_\delta(u) \operatorname{tr}(e^{iuZ}) \, du.
\]

For the unperturbed operator $Z = -i \frac{d}{d\tau}$ on $L^2(\mathbb{R})$, the trace is formally infinite, but the smoothed resolvent is regularized. The regularized trace yields
\[
\log D(s) \sim -\int_{-\infty}^{\infty} e^{-u} \log(1 - e^{-u - is}) \, du,
\]
which can be computed using contour integration and the Mellin transform. This integral evaluates to
\[
-\frac{1}{2}\psi(s/2) + \frac{1}{2}\log \pi + O(e^{-c|t|}),
\]
matching the known asymptotic expansion of $\log \Xi(s)$ \cite{IK2004}.

The full technical details are given in Appendix B, using the operator theoretic framework of Simon \cite{simon2005} and the explicit formula for the digamma function.
\end{proof}

\subsection{Explicit Constants via Resolvent Analysis}

We now provide explicit constants for the growth bound.

\begin{proposition}[Explicit Growth Constant]\label{prop:explicit-constant}
For $s = \sigma + it$ with $1/4 \leq \sigma \leq 3/4$ and $|t| \geq 1$, the canonical determinant satisfies
\[
|D(\sigma + it)| \leq e^{10} \cdot e^{2|t|}. \quad \text{(The constant $e^{10}$ is derived from explicit Schatten norm and resolvent estimates; see Appendix~C.)}
\]
% The constant $e^{10}$ is obtained from the explicit Schatten norm and resolvent estimates detailed in Appendix~C.
\end{proposition}

\begin{proof}
From the trace-class norm bound in Step 2 of Theorem \ref{thm:growth-bound}, and using the explicit Schatten estimates from Appendix C, we have
\[
\|B_\delta(s)\|_1 \leq \sum_{p \text{ prime}} C \frac{\log p}{p^2} \cdot \|R_\delta(s; Z)\|_{\text{op}}.
\]
The sum over primes converges to $C_1 \approx 0.45$ (related to $-\zeta'(2)/\zeta(2)$). The resolvent norm is bounded by $\|R_\delta(s; Z)\|_{\text{op}} \leq C_2 e^{C_3|\sigma - 1/2|}$ for appropriate constants $C_2, C_3$ determined by the smoothing kernel $w_\delta$.

Combining these estimates with the Fredholm determinant expansion and using the functional equation to control the boundary behavior, we obtain the stated bound.
\end{proof}

\subsection{Comparison with Classical Results}

\begin{remark}[Connection to Levin and Kurokawa/Fesenko]
The order 1 property of $D(s)$ aligns with the classical results on entire functions of exponential type \cite{levin1996}. The asymptotic analysis reproduces the behavior of adelic zeta functions studied by Kurokawa and Fesenko \cite{fesenko2021}, confirming that our construction is consistent with the expected properties of $\Xi(s)$ without assuming its Euler product form.
\end{remark}

\subsection{References for This Section}

The key technical results used in this section are:
\begin{itemize}
\item \textbf{Simon \cite{simon2005}}: Trace ideals and Schatten class estimates for operator perturbations.
\item \textbf{Levin \cite{levin1996}}: Distribution of zeros of entire functions and order estimates.
\item \textbf{Kurokawa/Fesenko \cite{fesenko2021}}: Adelic analysis and growth properties of zeta functions.
\item \textbf{Iwaniec–Kowalski \cite{IK2004}}: Analytic number theory and Phragmén–Lindelöf principles.
\end{itemize}
