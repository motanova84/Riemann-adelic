\section{Introduction}
\label{sec:introduction}

The Riemann Hypothesis (RH), formulated by Bernhard Riemann in 1859, asserts that all non-trivial zeros of the Riemann zeta function $\zeta(s)$ lie on the critical line $\Re(s) = 1/2$. This conjecture has profound implications for the distribution of prime numbers and remains one of the most important unsolved problems in mathematics.

\subsection{Historical Context}

Classical approaches to the Riemann Hypothesis have relied on the explicit formula connecting the zeros of $\zeta(s)$ to the distribution of primes:
\[
\psi(x) = x - \sum_{\rho} \frac{x^\rho}{\rho} - \frac{\zeta'(0)}{\zeta(0)} - \frac{1}{2}\log(1-x^{-2}),
\]
where $\psi(x) = \sum_{n \leq x} \Lambda(n)$ is the Chebyshev function, and the sum runs over non-trivial zeros $\rho$ of $\zeta(s)$.

Traditional methods have explored various approaches including:
\begin{itemize}
\item \textbf{Analytic continuation}: The functional equation and analytic properties of $\zeta(s)$
\item \textbf{Operator theory}: Hilbert-Pólya conjecture relating zeros to eigenvalues
\item \textbf{Dynamical systems}: Connections to ergodic theory and trace formulas
\item \textbf{Adelic methods}: Global-local principles from algebraic number theory
\end{itemize}

\subsection{Our Approach: S-Finite Adelic Spectral Systems}

This work introduces a fundamentally new framework based on \textbf{S-finite adelic spectral systems}. The key innovation is to construct a canonical entire function $D(s)$ using only operator-theoretic principles, without assuming:
\begin{enumerate}
\item The Riemann zeta function $\zeta(s)$ as input
\item The Euler product formula
\item The functional equation of $\zeta(s)$
\end{enumerate}

Instead, we build $D(s)$ from a spectral determinant arising from:
\begin{itemize}
\item A finite set $S$ of places (including the archimedean place)
\item Local operators at each place $v \in S$
\item A global trace formula emerging from geometric orbit structure
\item Spectral flow governed by canonical length scales $\ell_v$
\end{itemize}

\subsection{Main Results}

Our main contributions are:

\begin{theorem}[Unconditional RH via $D(s)$]
\label{thm:main_rh}
The canonical determinant $D(s)$ constructed from the S-finite adelic system satisfies:
\begin{enumerate}
\item $D(s)$ is entire of order $\leq 1$
\item $D(1-s) = D(s)$ (functional equation from spectral symmetry)
\item $D(s) \equiv \Xi(s)$ where $\Xi(s) = \frac{1}{2}s(s-1)\pi^{-s/2}\Gamma(s/2)\zeta(s)$
\item All zeros of $D(s)$ lie on $\Re(s) = 1/2$
\end{enumerate}
Therefore, all non-trivial zeros of $\zeta(s)$ lie on the critical line.
\end{theorem}

\begin{theorem}[Emergence of Local Lengths]
\label{thm:local_lengths}
The local length scales $\ell_v = \log q_v$ emerge geometrically from closed orbit structure, without input from $\zeta(s)$. This is proven via:
\begin{itemize}
\item Tate's theorem on local Fourier analysis
\item Weil's classification of closed orbits
\item Birman-Solomyak trace bounds
\end{itemize}
\end{theorem}

\subsection{Structure of This Paper}

The paper is organized as follows:
\begin{itemize}
\item \textbf{Section~\ref{sec:preliminaries}}: Adelic preliminaries and S-finite systems
\item \textbf{Section~\ref{sec:local_length}}: Geometric emergence of $\ell_v = \log q_v$
\item \textbf{Section~\ref{sec:hilbert_space}}: Construction of the spectral Hilbert space
\item \textbf{Section~\ref{sec:operator_resolvent}}: Operator theory and resolvent analysis
\item \textbf{Section~\ref{sec:functional_equation}}: Derivation of the functional equation
\item \textbf{Section~\ref{sec:growth_order}}: Growth estimates and order bounds
\item \textbf{Section~\ref{sec:pw_uniqueness}}: Paley-Wiener uniqueness theorem
\item \textbf{Section~\ref{sec:inversion_primes}}: Prime number inversion formula
\item \textbf{Section~\ref{sec:numerics}}: Numerical validation
\item \textbf{Section~\ref{sec:bsd}}: Extension to BSD conjecture (conditional)
\item \textbf{Section~\ref{sec:limitations}}: Limitations and open questions
\end{itemize}

Appendices provide detailed technical results on trace-class convergence, de Branges theory, Paley-Wiener multiplicities, archimedean contributions, computational algorithms, and reproducibility guidelines.

\subsection{Transparency and Reproducibility}

All code, data, and numerical validations are openly available at:
\begin{center}
\texttt{https://github.com/motanova84/-jmmotaburr-riemann-adelic}
\end{center}

This work is submitted for rigorous peer review with complete transparency regarding all assumptions, constructions, and computational verifications.
