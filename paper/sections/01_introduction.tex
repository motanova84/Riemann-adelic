\section{Introduction}

\subsection{Context and Motivation}

The Riemann Hypothesis (RH), formulated in 1859, remains one of the most fundamental open problems in mathematics. It asserts that all non-trivial zeros of the Riemann zeta function $\zeta(s)$ lie on the critical line $\Re(s) = 1/2$. Beyond its intrinsic mathematical interest, RH has profound implications for the distribution of prime numbers and connects to numerous areas of pure and applied mathematics.

Traditional approaches to RH have relied heavily on complex analysis, analytic number theory, and the properties of $\zeta(s)$ itself. In contrast, this work presents a \textbf{spectral-theoretic framework} rooted in adelic geometry, where $\zeta(s)$ emerges as a consequence rather than an assumption.

\subsection{Main Strategy}

Our approach consists of five interconnected steps:

\begin{enumerate}
  \item \textbf{Adelic Spectral Construction}: We build a canonical entire function $D(s)$ from first principles using S-finite adelic systems, without assuming the existence or properties of $\zeta(s)$.
  
  \item \textbf{Geometric Trace Formula}: Through Tate's theory and Weil's orbit identification, we derive that the orbit lengths are $\ell_v = \log q_v$ as a geometric consequence, not an axiom.
  
  \item \textbf{Functional Equation}: Using adelic Poisson summation with the Weil index, we prove that $D(1-s) = D(s)$.
  
  \item \textbf{Paley-Wiener Uniqueness}: We establish that $D(s) \equiv \Xi(s)$, where $\Xi(s)$ is the completed Riemann xi-function, using a strengthened uniqueness theorem with multiplicities.
  
  \item \textbf{Zero Localization}: We prove that all zeros lie on $\Re(s) = 1/2$ using two independent routes: de Branges canonical systems and Weil-Guinand positivity criteria.
\end{enumerate}

\subsection{Key Innovations}

This framework distinguishes itself through:

\begin{itemize}
  \item \textbf{Autonomy}: The construction of $D(s)$ does not presuppose $\zeta(s)$ or its Euler product.
  
  \item \textbf{Operator-Theoretic Foundation}: All analytic properties emerge from trace-class operator theory (Birman-Solomyak) and spectral geometry.
  
  \item \textbf{Adelic Naturality}: The S-finite restriction provides both convergence and a natural regularization mechanism.
  
  \item \textbf{Dual Zero Localization}: Two independent proofs ensure robustness and provide complementary insights.
\end{itemize}

\subsection{Structure of the Paper}

Section 2 establishes the adelic preliminaries and Schwartz-Bruhat theory. Section 3 derives the local length formula $\ell_v = \log q_v$ from Tate-Weil theory. Sections 4-5 construct the Hilbert space framework and define the resolvent operators. Section 6 proves the functional equation. Sections 7-8 establish growth estimates and Paley-Wiener uniqueness. Section 9 provides the inversion formula recovering primes. Section 10 presents numerical validation. Section 11 sketches the BSD extension (conditional). Section 12 discusses limitations and open questions.

The appendices provide detailed technical proofs: trace-class estimates via double operator integrals (A), de Branges canonical systems (B), Paley-Wiener theory with multiplicities (C), archimedean contributions (D), computational algorithms (E), and reproducibility protocols (F).
